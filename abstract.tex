The world trade organization (WTO) legally regulates the world trade system with its dispute settlement body (DSB).
While there exists a shared understanding among legal experts about how articles of WTO agreements interact with each other,
% Those understanding has been exclusively shared among the group of experts.
the complexity in these agreement-level networks has greatly constrained scholars as well as practitioners with limited legal resources from fully utilizing the dispute settlement mechanisms. 
% However, the complexity of the WTO legal framework has constrained many developing countries with limited legal knowledge and resources from fully utilizing the WTO DSB.
To address this issue, I propose a new method that effectively summarizes the systematic interactions between articles of WTO agreements.
I collect past 20 years of WTO disputes and trained a neural network that mimics the reasoning process of legal experts that determines which articles to cite for given factual description of the dispute.
I then investigated all the predictions from the trained neural network and fitted the summarization network using Random Forest.
Finally, I verify the quality of the fitted network by checking that the network captures the important systematic interactions as explained by the Panel and Appellate Body, two main judicial authorities of the WTO DSB.  %with its own l.%, Panel and Appellate Body. 
I find three major clusters of WTO DSB along three principles: Market Access, Reciprocity, and Non-discrimination.

% The pattern of interactions are shown as explained in the Panel and Appellate Body reports of the WTO DSB. 

