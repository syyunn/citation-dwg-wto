The world trade organiztion (WTO) legally regulates the world trade system with its dispute settlement body (DSB).
% Multiple articles of WTO agreements cooperate together and construct regulatory system of WTO DSB, 
% such as acheiving main principles of WTO and identifying the legal characterstics of a trade policy.
Though there exists an understanding among legal experts about how articles of WTO agreements interact with each other, % the entire map of interconnectedness between the articles of WTO agreements, 
however, those understanding has been exclusively shared among the group of experts.  %because it was hard to be materialized.
This exclusiveness has widen the gap
between the developing and developed countries in terms of
legal capacity and deteriorates the effectiveness of WTO DSB severely.
To address this issue, I propose a new method that materializes the relationship between articles of WTO agreements. %in a form of directed edge weight of the network.
I collected past 20 years of WTO disputes and trained a neural network that mimics the reasoning process of legal experts determining which articles to cite for given factual circumstances of trade dispute.
Then I collected all the predictions from the trained neural network and fitted a network of legal articles using a machine learning technique called \textit{Random Forest}.
I checked the fitted network captures the important interactions between articles closely to the authoritative judicial opinions of the WTO DSB for several important topics of trade, such as \textit{Market Access}, \textit{Reciprocity} and \textit{Regional Trade Agreement}.

% By defining the regulatory system of WTO DSB as a network of articles of WTO DSB, 




% and now has become the main agenda of WTO reform to give a satifsfa.
% researchers has selectively studied the relationship between a few number of articles. 


% the way of materialzing this interconnection has 
% addressed how to materialize those relationship as a whole.

% however, relationship between articles has never been studied as a whole.


% To address this issue, this paper suggests a new method 

% in a form of edge weight 


% find this relationship between articles in a form of \textit{weighted directed graph}.



% Members cite multiple articles to describe the legal identity of an illegal trade policy and claim its impaired benefit related to the WTO agreements.
% Since this co-citation pattern substantializes the regulatory system of WTO DSB about which set of articles of WTO agreements 
% cooperates to achieve main principles of WTO such and cover certain trade issue,



% WTO DSB adjudicates 
% over this citation and regulatory system of WTO DSB is substantialized.




% Multiple articles of WTO agreements interact with each other 
% to guarantee the keepin important principles of WTO such as \textit{Maket Access} or \textit{Reciprocity}.
% or to handle 

% interact with each other to achieve main principles of WTO agreements 



% The dispute settlement body of the World Trade Organization (WTO) maintains 



% Regulatory system of WTO DSB is notorious for its complexity because
% its legal system is designed to evolve 
% with a case law system. This complex nature