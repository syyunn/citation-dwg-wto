Assuring the \textit{Market Access} principle 
by enforcing
the binding tariff to imported products
, preventing the internal regulations 
and taxtions from discrimintating the domestic and imported products
and eliminating the quantitative restrictions on the border 
is a basic ideas to smooth the world trade flow. 

However, members are allowed to apply higher duties or to maintain a quantitative restrictions in certain circumstances, such as 
to act against dumping, to offset the subsidies 
and to take an emergency measure in case irrevisible injury to domestic industry is expected. Each of these examples 
are called \textit{Anti-dumping (AD)}, \textit{Countervailing Duties (CVD)} and \textit{Safeguard (SG)} respectively and recognized as 
three major types of \textit{Non-Tariff Barriers} that can constitue illegal trade barriers 
since these three measures often fail to satisfy the legal requirements of the WTO agreements and evolve into trade dispute in WTO DSB.

% Since these three types of measures 