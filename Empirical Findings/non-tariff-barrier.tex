\begin{figure}[h]
    \centering{
        \begin{tikzpicture}[
  >={Stealth[color=black]}
  ,shorten >=1pt,node distance=2cm
  ,on grid,initial/.style={}
  ,every label/.style={align=left}
]

  \linespread{2} % to adjust the line spacing inside the label

  \node[state, label=right:{ \hspace{4mm} Nullification or Impairment of \\[1mm] \hspace{0mm} any Benefit under the WTO Agreements}] (T1) {XXIII};

  % \node[state, label=right:{Interpretation of \\[1mm] \hspace{-1mm} Tariff Concession}] (T1) {II:1};
  % \node[state, label=left:{General Elimination of \\[1mm] \hspace{-3mm} Quantitative Restrictions}] at ([shift=({240:3 cm})]T1) (T4) {XI:1};
  % \node[state, label=right:{Fair Adminstration of \\[1mm] \hspace{-1mm} Laws and Regulations}] at ([shift=({240:3 cm})]T4) (T5) {X:3(a)};
  % \node[state, label=above:{Non-discriminatory Administration of \\[1mm] \hspace{10mm} Quantitative Restrictions}] at ([shift=({100:6 cm})]T4) (T7) {XIII:1};
  % \node[state, label=right:{National Treatment on \\[1mm] \hspace{3mm} Internal Taxation}] at ([shift=({10:2.5 cm})]T4) (T6) {III:2};
  % % \node[state, label=above:Fees Connected with Importation and Exportation] at ([shift=({150:5.5 cm})]T4) (T8) {VIII};
  % % \node[state, label=right:{National Treatment on \\[1mm] \hspace{2mm} Internal Taxation}] at ([shift=({0:5 cm})]T1) (T2) {$v_j$ = III:2};
  % \node[state, label=left:{Fees Connected with \\[1mm] \hspace{-10mm} Importation and Exportation}] at ([shift=({150:5.5 cm})]T4) (T8) {VIII};


  % \begin{scope}[every edge/.append style={-, double=black, draw=white}] % for directed edge, change "style={->, double=black, draw=white}]"
  %   \path (T1)
  %   edge   (T4)
  %   edge   (T6);
  %   \path (T4)
  %   edge   (T5)
  %   edge   (T6)
  %   edge   (T7);
  %   \path (T5)
  %   edge   (T6);
  %   \path (T8)
  %   edge   (T7)
  %   edge   (T1)
  %   edge   (T6);

  % \end{scope}
\end{tikzpicture}

% to draw the node's border w/ color, refer to https://tex.stackexchange.com/questions/438412/how-to-add-border-to-a-node
    }
    \caption{{\bf Network of Articles that Achieves \textit{Reciprocity} for Non-Tariff Barriers:}
        This figure demonstrates a network of articles of WTO agreements
        that cooperatively regulates \textit{Non-Tariff Barriers} (NTBs) in WTO DSB.
        Three major NTBs such as \textit{Anti-dumping (AD)}, \textit{Countervailing Duties (CVD)}
        and \textit{Safeguard (SG)} are relying on the Article XXIII to resolve its inconsistency with the principle of \textit{Reciprocity}.
        %(Directions and weights of network edges are omitted for brevity.)
    }
    \label{fig:ntb-explained}
\end{figure}
 
Ensuring the \textit{Market Access} principle
% by enforcing
% the binding tariff to imported products, by preventing the internal regulations from discriminating the domestic and imported products
% and by eliminating the quantitative restrictions
is a basic approach to smooth the world trade flow as explained in the previous section. %subsection \ref{emp:ma}.
However, members are sometimes allowed to apply higher duties or to maintain a quantitative restrictions in certain circumstances, such as
to act against low-price dumping of foreign producers, to offset the distorted competitive relationship resulted from the illegal subsidies, or to take an emergency measure in case irreversible injury to the domestic industry is expected. Each of these examples
are called \textit{Anti-dumping (AD)}, \textit{Countervailing Duties (CVD)} and \textit{Safeguard (SG)} respectively. These measures are recognized as
three major types of \textit{Non-Tariff Barriers} (NTBs) %that can easily constitute illegal trade barriers
since these three measures often fail to satisfy the legal requirements of the WTO agreements and evolve into trade disputes.
 
The principle of \textit{reciprocity} is a general notion that regulates these three major NTBs.
This principle requires a change of the value of imports affected by a country's trade policy
shall be balanced with the equal value of exports across trading partners affected \citep{bagwell1999}.
The rules of the WTO agreements, in particular Article XXIII of GATT 1994,
confers a right to insist on its nullified or impaired benefit that is expected under the WTO agreements and require satisfactory adjustment to the member
who results in such nullification or impairment. Moreover, this article confers a right to retaliate if no satisfactory adjustment is being fulfilled.
 
Therefore, Article XXIII regulates this action-reaction characteristics of \textit{AD}, \textit{CVD} and \textit{SG} to achieve the principle of \textit{reciprocity}. % as shown in Figure \ref{fig:ntb-explained}.
For example, Panel stated that a member can resort to Article XXIII and raise a legal claim in case another member levies unjustified antidumping duties that fails to fully explain the causal relationship between the dumping and the material injury to the related industry in its report on \textit{New Zealand - Imports of Electrical Transformers from Finland}\footnote{\textit{WTO official document} L/5814, adopted on 18 July 1985, pp.67-68, para.4.4.}:
 
\begin{displayquote}[]
   ``Panel believed that if a contracting party
   affected by the determination (of levying \textbf{anti-dumping duties}) could make a case that the importation could not in itself have the effect of
   causing material injury to the industry in question, that contracting party was entitled, under the relevant
   GATT provisions, \textbf{in particular Article XXIII}, that its representations be given sympathetic consideration
   and that eventually, \textbf{if no satisfactory adjustment was effected, it might refer the matter to the CONTRACTING
   PARTIES\footnote{It means raising a legal claim through WTO DSB}}, as had been done by Finland in the present case."
\end{displayquote}
 
\noindent This relationship is captured by the $6.3\%$ weight of directed edge from Article VI to Article XXIII. This is because citing Article VI (Antidumping duties and CVD) naturally leads to the citation of Article XXIII to insist its impaired benefit and solicit the satisfactory adjustment relying on the principle of \textit{reciprocity} in WTO DSB.
Also, other two types of measures, \textit{Illegal Subsidy} and \textit{SG} whose legal requirements are regulated under the Article XVI and XIX respectively also heavily rely ($>$ 10\%) on the Article XXIII to solicit its impaired benefit and satisfactory adjustment based on the principle of \textit{reciprocity} in case the breach of those requirements occurs.
 
% Since these three types of measures
 

