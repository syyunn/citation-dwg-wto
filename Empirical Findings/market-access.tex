
% == THREE SUBSYSTEM == 
\begin{figure}
    \centering{
        \input{market_access_directed.tex}
    }
    \caption{{\bf Network of the Articles that Achieves \textit{Market Access}:}
        This figure demonstrates a network of articles of WTO agreements
        that cooperatively achieves the principle of \textit{Market Access} in the regulatory system of WTO.
        Tariff and Non-tariff barriers such as quantitative restriction, internal taxations
        and extra fees for crossing border can inhibit the chance of foreign goods to access the foreign market.
        Therefore, these articles tend to work together to ensure the \textit{Market Access} principle working properly.
        %(Directions and weights of network edges are omitted for brevity.)
    }
    \label{fig:market-aceess_directed}
\end{figure}

Figure \ref{fig:market-aceess_directed} shows a part of the entire network that explains how
the articles of WTO agreements co-cite each other to acheive one of the main principles of 
WTO, market access. Market access 
refer to the guarantee of the conditions of
tariff and non-tariff measures 
agreed by members for the entry of specific 
goods into their markets.
