 
% == THREE SUBSYSTEM ==
\begin{figure}[ht]
    \centering{
        \begin{tikzpicture}[>={Stealth[color=black]},shorten >=1pt,node distance=2cm,on grid,initial/.style={}]
  \node[state, label=right:Interpretation of Tariff Concession] (T1) {II:1};
  \node[state, label=left:General Elimination of Quantitative Restrictions] at ([shift=({240:3 cm})]T1) (T4) {XI:1};
  \node[state, label=right:Fair Adminstration of Laws and Regulations] at ([shift=({240:3 cm})]T4) (T5) {X:3(a)};
  \node[state, label=above:Non-discriminatory Administration of Quantitative Restrictions] at ([shift=({100:6 cm})]T4) (T7) {XIII:1};
  \node[state, label=right:National Treatment on Internal Taxation] at ([shift=({10:2.5 cm})]T4) (T6) {III:2};
  \node[state, label=above:Fees Connected with Importation and Exportation] at ([shift=({150:5.5 cm})]T4) (T8) {VIII};

  \begin{scope}[every edge/.append style={->, double=black, draw=white}] % for directed edge, change "style={->, double=black, draw=white}]"
    \path (T1)
    edge   (T4)
    edge   (T6);
    \path (T4)
    edge   (T5)
    edge   (T6)
    edge   (T7);
    \path (T5)
    edge   (T6);
    \path (T8)
    edge   (T7)
    edge   (T1)
    edge   (T6);

  \end{scope}
\end{tikzpicture}

% to draw the node's border w/ color, refer to https://tex.stackexchange.com/questions/438412/how-to-add-border-to-a-node
    }
    \caption{{\bf Network of the Articles that Achieves \textit{Market Access}:}
        This figure demonstrates a network of articles of WTO agreements
        that cooperatively achieves the principle of \textit{Market Access} in the regulatory system of WTO.
        Tariff and Non-tariff barriers such as quantitative restriction, internal taxations
        and extra fees for crossing borders can inhibit the chance of foreign goods to access the foreign market.
        Therefore, these articles tend to work together to ensure the \textit{Market Access} principle working properly.
        %(Directions and weights of network edges are omitted for brevity.)
    }
    \label{fig:market-aceess_directed}
  \end{figure}
  % Figure \ref{fig:market-aceess_directed} shows a part of the fitted network $G^*$ that explains how
 % the articles of WTO agreements are interconnected each other to achieve one of the main principles of
 % WTO, \textit{Market Access}.
 \textit{Market Access}
 refers to the guarantee of the conditions relating to the
 tariff or non-tariff measures
 for the entry of
 goods into the foreign market. For example, if
 a country set a limit on the quantity of imported goods, no foreign goods can get access to that market above that limit. Also, if a country sets too high tariff rate for the foreign goods that cross the border,
 it can prevent those foreign goods from accessing that market.
 Therefore, \textit{Market Access} is the most basic but the most important principle of the WTO
 because it directly represents the primary goal of the WTO.
  
 Three different types of measures - \textit{quantitative restriction, tariff} and \textit{internal regulations} - are mostly preferred by the countries to protect their domestic market from the importation of foreign goods. For example, a country can
 discriminate imported and domestic products with its internal regulation as shown in the \textit{Korea - Beef} case in Section \ref{subsec:design:dnn}.
 % This discrimination provides an unfair ground of competition between domestic and imported products and it eventually excluded imported beef from the Korea beef market.
 These three types of measures that are mostly relevant to the \textit{Market Access} are illustrated in Figure \ref{fig:market-aceess_directed}.
 Article XI:1, II:1 and III correspond to the obligations that regulate quantitative restriction, tariff and internal regulations respectively.
 Edges are colored in red if the mean of the two directed edge weights are greater than $10\%$. For example, $w^*(\text{II:1}, \text{III}) = 0.117$ and $w^*(\text{III}, \text{II:1}) = 0.122$ thus their simple mean is 11.95\% which is greater than 10\%.
 This red triangle that is comprised of Article XI:1, II:1 and III shows that the fitted network $G^* = (V, E, W^*)$ captures the principle of \textit{Market Access} as a cluster of these three articles with relatively large edge weights ($\: >10\%$)\footnote{Compare $10\%$ to the simple mean of total edge weight $100\%$ over all 79 possible source nodes, $1.26\%$ $(\sim 100\% / 79)$}.
 These three articles are \textit{cohesively} interconnected because they need each other to clearly distinguish the role of each one from others to achieve the \textit{Market Access}.
 For example, \textit{Panel} explicitly opinionated that Article XI:1 and Article III shall take in charge of \textit{quantitative restriction} and \textit{internal requirements} respectively without duplication of their responsibilities as following:
  
 \begin{displayquote}[][]
 ``the General Agreement distinguishes between measures affecting the `importation' of products,
 which are regulated in Article XI:1, and those affecting `imported products', which are dealt with in
 Article III. \textbf{If Article XI:1 were interpreted broadly to cover also internal requirements, Article III
 would be partly superfluous.}"
 \end{displayquote}
  
 In addition to it,
 the fitted network $G^*$ also captures sub-articles that support the \textit{Market Access} triangle. For example, I colored edges in blue if the mean of the two directed edge weights are greater than 5\% and less than 10\%.
 The Article X:3(a) which prohibits \textit{unfair administration of laws and regulations} are connected to the Article XI:1 and III because those two are mostly
 deteriorates the principle of \textit{Market Access} by administering their relevant laws and regulations unfairly. Also, in a similar notion, Article VIII that prohibits \textit{the collection of abnormal amount of fees connected with importation or exportation} is connected to the Article II:1 because
 often those fees are indistinguishable from the tariff schedule whose Interpretation is regulated under Article II:1. For example, in \textit{Argentina - Textile} case, Argentina insisted that they didn't collect fees connected to importation inconsistently with the obligation under Article VIII because they included the fees into the tariff schedule. %which is regulated by the Article II.
 However, \textit{Panel} refuted this logic as following\footnote{Panel Report, Argentina – Textiles and Apparel, paras. 6.81-6.82.}:
  
 \begin{displayquote}[][]
 ``The provisions of the WTO Understanding on the Interpretation of \textbf{Article II:1(b)} of
 GATT 1994, dealing with `other duties and charges’, make clear that \textbf{including a
 charge in a schedule of concessions in no way immunizes that charge from challenge
 as a violation of an applicable GATT rule} \ldots Therefore, we consider that the fact that Argentina's statistical tax is included in its
 Schedule is not a defence to its inconsistency with the provisions of \textbf{Article VIII} of
 GATT."
 \end{displayquote}
  
 % \noindent These supportive relationships are captured in $W^*$ with relatively high $(>5\%)$ weights but lower than the most important triangle.
  
 % ligation under the Article VIII in \textit{Argentina - Textile} case,
  
 % Mean of directive edge weights between Article II:1, III and XI:1 are $0.289$,
  
  
  
 % having a sum of directive edge weights bigger than 0.15, which is about ten times bigger than that of 
  
 % The three major articles of the WTO agreements
 % that regulates
  
  
 
 