
% == THREE SUBSYSTEM == 
\begin{figure}[ht]
    \centering{
        \input{market_access_directed.tex}
    }
    \caption{{\bf Network of the Articles that Achieves \textit{Market Access}:}
        This figure demonstrates a network of articles of WTO agreements
        that cooperatively achieves the principle of \textit{Market Access} in the regulatory system of WTO.
        Tariff and Non-tariff barriers such as quantitative restriction, internal taxations
        and extra fees for crossing border can inhibit the chance of foreign goods to access the foreign market.
        Therefore, these articles tend to work together to ensure the \textit{Market Access} principle working properly.
        %(Directions and weights of network edges are omitted for brevity.)
    }
    \label{fig:market-aceess_directed}
  \end{figure}
  
% Figure \ref{fig:market-aceess_directed} shows a part of the fitted network $G^*$ that explains how
% the articles of WTO agreements are interconnected each other to acheive one of the main principles of 
% WTO, \textit{Market Access}. 
\textit{Market Access} 
refers to the guarantee of the conditions relating to the 
tariff or non-tariff measures 
for the entry of 
goods into the foreign market. For example, if 
a country set a limit on the quantity of imported goods, no foreign goods above that limit can get access to that market. Also, if a country set a high tariff rate for a imported goods to cross the border, 
it also prevents foreign goods from accessing to that market.
Therefore, \textit{Market Access} is the most basic but important principle of the WTO
because it directly represent the primary goal of the WTO, smoothing the world trade flow.

Three different types of measures - quantitative restriction, tariff and internal taxation or regulations - are preferred by countries to protect the importation of certain product. For example, a country can
discriminate imported and domestic product with its internal regulation such as enforcing the specialized retail channel for the imported product as exmplified in the \textit{Korea - Beef} case (\textit{See}\ref{subsec:design:dnn}).
This discrimination provides a unfair ground of competition between domestic and imported products and it eventually excluded imported beef from the Korean beef market.
These three main types of measures that are relevant to the \textit{Market Access} is illustrated in Figure \ref{fig:market-aceess_directed}.
Article XI:1, II:1 and III correponds to the obligations that regulates quantitative restriction, tariff and internal regulations respectively.
Edges are colored in red if the mean of the two directed edge weights are greater than $10\%$. For example, $w^*(\text{II:1}, \text{III}) = 0.117$ and $w^*(\text{III}, \text{II:1}) = 0.122$ thus their simple mean is 11.95\% which is greater than 10\%.
This red triangle that is comprised of Article XI:1, II:1 and III shows that the fitted network $G^* = (V, E, W^*)$ captures the principle of \textit{Market Access} as a cluster of these three articles with relatively large edge weights ($\: >10\%$) compared to the simple mean of edge weight, $100\% / 79 \sim 1.26\%$. 
These three articles are cohesively interconnected because they need each other to clealry determine the role of each one to achieve the \textit{Market Access} principle.
For example, \textit{Panel} explained the relationship between Article XI:1 and III in \textit{India - Autos} case as following\footnote{Panel Report, India – Autos, para. 7.220.}:

\begin{displayquote}[][]
``the General Agreement distinguishes between measures affecting the `importation' of products,
which are regulated in Article XI:1, and those affecting `imported products', which are dealt with in
Article III. \textbf{If Article XI:1 were interpreted broadly to cover also internal requirements, Article III
would be partly superfluous.}"
\end{displayquote}




In addition to it, 
the fitted network $G^*$ also captures sub-articles that supports this \textit{Market Access} triangle. For example, I colored edges in blue if the mean of the two directed edge weights are greather than 5\% and less than 10\%.
The Article X:3(a) which prohibits a unfair adminstration of laws and regulations are connected to the Article XI:1 and III that are mostly 
deteriorates the principle of \textit{Market Access} by unfair adminstration of laws and regulations relating to the quantitative restriction 
and competitive relationship between imported and domestic products. Also, in a similar notion, Article VIII get connected to the Article II:1 because 
often abnormal amount of fees connected with the importation are charged and it deteriorates the principle of \textit{Market Access} as well. For example, in \textit{Argentina - Textile} case, Argentina insisted that they didn't collect fees connected to importation inconsistently with the obligation under Article VIII because they included the fees into the tariff schedule which is regulated by the Article II.
However, \textit{Panel} refuted this logic as following\footnote{Panel Report, Argentina – Textiles and Apparel, paras. 6.81-6.82.}:

\begin{displayquote}[][]
``The provisions of the WTO Understanding on the Interpretation of Article II:1(b) of
GATT 1994, dealing with `other duties and charges’, make clear that \textbf{including a
charge in a schedule of concessions in no way immunizes that charge from challenge
as a violation of an applicable GATT rule} \ldots Therefore, we consider that the fact that Argentina's statistical tax is included in its
Schedule is not a defence to its inconsistency with the provisions of Article VIII of
GATT."
\end{displayquote}


% ligation under the Article VIII in \textit{Argentina - Textile} case, 

% Mean of directive edge weights between Article II:1, III and XI:1 are $0.289$, 



% having a sum of directive edge weights bigger than 0.15, which is about ten times bigger than that of  

% The three major articles of the WTO agreements 
% that regulates 
