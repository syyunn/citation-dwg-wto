The term \textit{Regional Trade Agreements} (RTAs)
refer to any reciprocal trade agreements between two or more trade partners.
WTO permits its members to enter into RTAs under specific conditions written in Article XXIV.
% as illustrated in
% Figure \ref{fig:rta-explained}.
However, Article XXIV roughly draws a basic condition to formulate RTAs and its
detailed regulatory system has evolved with a number of disputes arise
from numerous discriminatory uses of RTA. For example,
in \textit{US - Line Pipe} case,
the United States applied its safeguard measures to its trade partners excluding Canada and Mexico who are members of the NAFTA although
the United States had included Canada and Mexico in the analysis of whether increased imports caused serious injury to domestic production.
Therefore, \textit{Appellate Body} concluded this discriminatory application of safeguard measures as inconsistent with the rules of the WTO agreements.
This case, where the breach of Article XIX (Safeguard) leads to the breach of Article XXIV (Non-discrimination between RTA and Non-RTA members)
is captured in the $12.7\%$ weight of directed edge from Article XIX to Article XXIV in Figure \ref{fig:rta-explained}.
This relatively high edge weight results from the numerous real-world cases where the members frequently discriminate non-RTA members from the RTA members when they maintain its \textit{Safeguard} measures to protect their domestic industry.
Also, we can understand the opposite direction of $8\%$ weighted edge from Article XXIV to Article XIX as the \textit{non-discrimination}
principle in Article XXIV is
extended to Article XIX as members of WTO have repeatedly used RTAs to justify their discriminatory application of \textit{Safeguard measures} between RTA and non-RTA members.%\footnote{It's worth noting that we have a total 79 source articles per target article. Therefore, naive baseline of the edge weight can be calculated by simple mean, $100\% / 79 \sim 01.26\%$}
 
In addition to the case of \textit{Safeguard},
members frequently try to justify their discriminatory
use of quantitative restriction by insisting that Article XXIV
has conferred its right to unequally distribute the quantitative restriction between RTA and Non-RTA members.
For example, in the \textit{Turkey – Textiles} case, Turkey defenced himself that Article XXIV authorizes
members forming a customs union to deviate from the obligations of Article XIII that requires members to apply quantitaitive restriction without discrimination between members.
However, \textit{Appellate Body} explicitly refuted this logic by saying\footnote{Appellate Body Report, Turkey – Textiles, para. 65.}:
\begin{displayquote}[][]
   ``Article XXIV does not allow Turkey to adopt, upon the
   formation of a customs union with the European Communities, quantitative restrictions \ldots \textbf{which
   were found inconsistent with Articles XI and XIII of GATT 1994} and Article 2.4 of the ATC.
   However, the Appellate Body stressed that it was only finding that \textbf{Turkey's quantitative
   restrictions at issue were not justified by Article XXIV} \ldots"
\end{displayquote}
This legal context that Article XXIV does not exempt the obligation under Article XIII is captured by the $8\%$ weighted directed edge from Article XXIV to Article XIII in Figure \ref{fig:rta-explained}.
 
% Quantitative restriction
 
% Moreover, \textit{non-discrimination} obligation under Article XIII shall be interpreted within the meaning of the Article XI according to a number of panel reports\footnote{Panel Report, \textit{US – Shrimp}, para.7.22;  Panel Report, \textit{India – Quantitative Restrictions}, para. 5.17, Panel Report in \textit{Colombia – Ports of Entry} para. 7.291.  }.
% However, jurisprudence in those panel reports allows a general quantitative restriction for special occasions of \textit{reciprocal} understanding like RTAs.
 
 
% which prohibits quantitative restriction in general without any special occasions of \textit{reciprocal} understanding like RTAs.
% For example,
 
% this context is captured in the directive weighted edge from \textit{XI} to \textit{XIII} in the same Figure. 
 

