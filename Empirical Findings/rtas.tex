The term \textit{Regional Trade Agreements} (RTAs)
refer to any reciprocal trade agreements between two or more trade partners. 
WTO permits its members to enter into RTAs under specific conditions written in the \textit{Article XXIV} 
as illustrated in 
Figure \ref{fig:rta-explained}.
However, \textit{Article XXIV} roughly draws a basic condition to formulate RTAs and its 
detailed regulatory system has evolved with a number of disputes arises 
from numerous discriminatory uses of the RTAs. For example,
in \textit{US - Line Pipe} case, 
the United States applied its safeguard measures to its trade partners excluding Canada and Mexico, who are members of the NAFTA, although 
the United States had included Canada and Mexico in the analysis of whether increased imports caused serious injury to domestic production.
Therefore, \textit{Appellate Body} concluded this discriminatory application of safeguard measures as inconsistent with the rules of the WTO agreements. 
Figure \ref{fig:rta-explained} 
captures this directive relationship 
where the breach of the \textit{Article XIX (Safeguard)} leads to the breach of the \textit{Article XXIV} 
where it requires members not to discriminate between non-RTA and RTA members while formulate and maintain RTAs with an $12.7\%$ weighted directive edge 
from \textit{XIX} to \textit{XXIV}. Also, we can understand the directive weighted edge from \textit{XXIV} to \textit{XIX} as \textit{non-discriminatory} 
principle in the \textit{Article XXIV} become 
extended to the \textit{Article XIX} as members of WTO have used RTAs to justify their discriminatory application of \textit{Safeguard measures} between RTA and non-RTA trade partners.\footnote{It's worth noting that we have total 79 source articles for each target article. Therefore, naive baseline of the edgeweight calculated by simple mean is $0.0126\%$}

In addition to the case of \textit{Safeguard}, 
members sometimes try to justify their discriminatory 
use of quantitaitive restriction on imported products. For example, in the \textit{Turkey – Textiles} case, Turkey defenced himself that \textit{Article XXIV} authorizes
members forming a customs union to deviate from the obligations of \textit{Article XIII} that requires members to apply quantitaitive restrction without discrimination between members. 
However, \textit{Appellate Body} explicitly refuted this logic as saying\footnote{Appellate Body Report, Turkey – Textiles, para. 65.}:
\begin{displayquote}[][]
    ``Article XXIV does not allow Turkey to adopt, upon the
    formation of a customs union with the European Communities, quantitative restrictions \ldots \textbf{\textit{which
    were found inconsistent with Articles XI and XIII of GATT 1994}} and Article 2.4 of the ATC.
    However, the Appellate Body stressed that it was only finding that \textbf{\textit{Turkey's quantitative
    restrictions at issue were not justified by Article XXIV}} \ldots"
\end{displayquote}
This legal context tha \textit{Article XXIV} does not exempt the obligation of the \textit{Article XIII} is captured by the directive weighted edge from the \textit{XXIV} to \textit{XIII} in Figure \ref{fig:rta-explained}.
