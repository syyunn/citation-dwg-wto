To map the regulatory system of WTO DSB with a network of legal articles of WTO agreement,
this paper introduces a novel design of deep neural network \citep{DBLP:journals/corr/Schmidhuber14} that
gets two different types of input at the same time.
One is textual description of trade policy that led to the dispute and
the other one is the content of the specific legal article of the WTO agreement.
This design is improvised to mimic
the reasoning process of the legal practitioners
where the legal practitioners read
the text description of
factual circumstances of the dispute and think about the contents of
the applicable legal articles at the same time.

\begin{itemize}
    \item data composition and collection
    \item w/ and wo/ textual information - horizontal comparison
    \item implication of this paper, application to solve real world problems
\end{itemize}


% Moreover, this paper empricially
% finds that it's important to explicitly
% consider the textual information
% to properly map the regulatory network of WTO DSB.
% By comparing the results of two different approaches,
% regulatory network reconstructed with and without textual information,
% this paper shows textual content that describes
% characteristics of each trade policy
 
% it's important

% to properly map
 
