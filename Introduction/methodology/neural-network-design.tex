
To train this neural network, this paper collected textual description of trade policy 
that led to the dispute and articles of the WTO agreement cited for each dispute
case requested to the WTO DSB 
from 1995 to 2018 (\hyperref[sub:cited-articles-table]{Total $143$ cases. \textit{Check} the list in Appendix A.2}).
Using this collected data, I trained the neural network by enforcing the neural network to answer correctly 
whether a given article of the WTO agreements
can be cited for the given textual description of 
trade policy. (\textit{See} Figure \ref{fig:design-of-nn} and Figure \ref{fig:def:io:nn}).

After training, I fitted a set of directed edge weights $W^*$ that 
best explains the variance of each article's citabilities that are predicted by the trained deep neural network using a machine learning technique \textit{Random Forests} \citep{rf, genie3}. 

After fitting $W^*$, to check whether this fitted network of articles of the WTO agreements\\  $G^*$ = ($V$, $E$, $W^*$) maps the jurisprudences of WTO DSB properly, this paper
compares the way of the fitted network $G^*$ explaining how articles of WTO agreements achieve some main principles of WTO, such as \textit{Market Access}, \textit{Reciprocity} and \textit{Non-discrimination} 
with the jurisprudences of \textit{Panel} and \textit{Appellate Body}. This comparison reveals that the fitted network $G^*$ captures the interactions between the articles of WTO agreements
closely to the jurisprudences of \textit{Panel} and \textit{the Appellate Body}. We can infer from this similarity that the fitted network $G^*$ closely maps the jurisprudences of WTO DSB. This is because those two judicial bodies 
authoritatively consitute the jurisprudences over how rules of WTO agreements are working together 
to achieve those main principles.

Finally, upon this similarity, this paper offers this methodology as
an alternative solution to the widening gap of legal capacity between developing and developed countries in WTO DSB.
Since this method effectively materializes the shared understanding of legal experts and reveals important interactions between articles inside the system of WTO DSB,
it can lower the cost to build the same amount of legal capacity to understand the WTO DSB.
Moreover, rather than keep relying on previous approach that provides legal advice to developing countries that does not create a shared understanding over the system between developing and developed countries,
if we shift our focus on how to materialize the current shape of the system, WTO
will become more effective as members being able to discuss their trade issues upon the measurable ground of shared understanding about how WTO works.

