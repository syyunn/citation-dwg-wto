To map the regulatory system of WTO DSB as 
a network of legal articles of WTO agreement 
successfully,
this paper designs a deep neural network (Figure \ref{fig:traditional-convolutional-network}) that
processes two different types of textual information.
% This deep neural network 
% processes two different types of input.
One is textual description of trade policy that led to the dispute (\hyperref[sub:factual-aspect-example]{Appendix A.1}) and
the other one is the content of a legal article of the WTO agreement (Figure \ref{fig:gatt_art1}).
This design is improvised to mimic
the reasoning process of WTO legal practitioners
where the legal practitioners read
the text description of
factual circumstances of the dispute and imagine regulatory contents of
the applicable legal articles simultaneously.


To train this neural network, this paper collected textual description of trade policy 
that led to the dispute and articles of the WTO agreement cited for each dispute
case requested to the WTO DSB 
from 1995 to 2018 (\hyperref[sub:cited-articles-table]{Appendix A.2}).
Using this collected data, I trained above mentioned neural network by enforcing the neural network to answer correctly 
about whether the given article of the WTO agreement
can be cited for the given textual description of 
trade policy that led to the dispute.

After finish training, I collected all the answers from trained neural network (Figure \ref{predidction_matrix})
and created a network of legal articles of the WTO agreement\footnote{Part of the created network is exemplified in the Figure\ref{fig:market-aceess_directed}} from this collection of answers using a machine learning technique called GENIE3 \citep{genie3} 
which is widely used in the biomedical engineering to reconstruct gene regulatory networks\footnote{Anaology of international normative system to genetics maybe natural because gene expressions (main principles of WTO) are governed by complex interaction between multiple regulatory proteins (legal articles of WTO). \textit{See also} \cite{gene_analogy}}.

To check whether this created network of WTO agreements maps the regulatory system of WTO DSB properly, this paper
compares the created network and the jurisprudence of WTO DSB made by Panel and Appellate Body. 
The comparison shows that the network captures the interaction between the articles of WTO agreements
as similar to the jurisprudence of the Panel and the Appellate Body. This similarity can be interpreted 
as the created network closely maps the regulatory system of WTO DSB since these two judicial bodies 
consitute authoritative interpreation over how rules of WTO agreements are working together to achieve
main principles of WTO.

Moreover, this paper justifies the use of neural network to process textual information by showing the limitation of not using the textual information 
to perform this mapping of regulatory system of WTO DSB. 

% By doing so, this paper introduces a new method 
% that maps a regulatory system of WTO DSB as a network of articles of WTO agreements using deep learning.




% contributes following points:
% \begin{itemize}
%   \item
%   \item
%   \item
% \end{itemize}
%using the collected answers from the neural network .



% \input{Figures/NN.tex}

% 
\begin{xltabular}{\linewidth}{ l | X }
    \caption{Description of Variables used in this Study} 
   \label{table: vardescription}\\
   \hline \hline
  
  \textbf{\normalsize Code} & \textbf{\normalsize Definition and source}  \\
   \hline 
  \endfirsthead
   \hline \hline
  
  \textbf{\normalsize Code} & \textbf{\normalsize Definition and source}  \\
   \hline 
  \endhead
  
  \textbf{exportsgr} & Exports of goods and services (annual \% growth) retrieved from World Bank. \\ \hline 
  
  
  \textbf{importsgr} & Imports of goods and services (annual \% growth) retrieved from World Bank.\\ \hline 
  
  
  \textbf{gr\_tot} & Terms of trade change over previous year (in \%).  Data for terms of trade are collected from theglobaleconomy.com  and Kaminsky and Reinhart online database. Since variables have two different base years, the base year for both was changed to 2000 to have the same base year. And then the change is calculated as below.  
  
  \begin{equation}
  gr\_tot_{i,t} = (\frac{tot_{i,t}- tot_{i,t-1}}{tot_{i,t-1}})100
  \end{equation}
  \\ \hline
  
  \end{xltabular}



% where nodes are the articles of the WTO agreement and the edges are ranking of importance between incoming nodes to each 
% target node using a tree-based ensemble machine learning method 
% using the answers of the trained neural network.
%, called GENIE3 \citep{genie3} 
