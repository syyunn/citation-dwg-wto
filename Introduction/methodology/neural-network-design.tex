To map the regulatory system of WTO DSB as 
a network of legal articles of WTO agreement 
successfully,
this paper designs a deep neural network (Figure \ref{fig:traditional-convolutional-network}) that
processes two different types of textual information.
One is textual description of the dispute (\textit{See} an example at \hyperref[sub:factual-aspect-example]{Appendix A.1}) and
the other one is the textual content of a legal article of the WTO agreements (\textit{See} an example at Figure \ref{fig:gatt_art1}).
This design is improvised to mimic
the reasoning process of WTO legal practitioners
where the legal practitioners read
the textual description of
factual circumstances of the dispute and imagine regulatory contents of
the applicable legal articles while he/she reads the factual description.


To train this neural network, this paper collected textual description of trade policy 
that led to the dispute and articles of the WTO agreement cited for each dispute
case requested to the WTO DSB 
from 1995 to 2018 (\hyperref[sub:cited-articles-table]{Total $143$ cases. \textit{Check} the list in Appendix A.2}).
Using this collected data, I trained the neural network by enforcing the neural network to answer correctly 
whether a given article of the WTO agreements
can be cited for the given textual description of 
trade policy that led to the dispute.

After finish training, I collected all the answers from the trained neural network (Figure \ref{predidction_matrix})
and fitted a network of legal articles of the WTO agreement
using this collection of answers.
Since a network $G$ comprises $V$ (set of articles), $E$ (directed edges) and their $w$ (weights) as defined in Figure \ref{fig:def}, 
I fitted a best set of weights $w^*$ for given $V$ and $E$
using a machine learning technique called GENIE3 \citep{genie3} 
which is widely used in the biomedical engineering to reconstruct gene regulatory networks.\footnote{Anaology of international normative system to genetics maybe natural because gene expressions (achieving main principles of WTO) are governed by complex interaction between multiple regulatory proteins (interaction between legal articles of WTO). Similar notion is adopted in \cite{gene_analogy} to explain the the evolution of norm of transparency in international security.}

To check whether this fitted network of WTO agreements $G^*$ maps the regulatory system of WTO DSB properly, this paper
compares the created network and the jurisprudence of WTO DSB made by \textit{Panel} and \textit{Appellate Body}. 
This comparison reveals that the fitted network $G^*$ captures the interaction between the articles of WTO agreements
similarly with the jurisprudence of \textit{Panel} and \textit{the Appellate Body}. This similarity guarantees that the fitted network $G^*$ closely maps the regulatory system of WTO DSB since only these two judicial bodies 
can authoritatively consitute the jurisprudence over how rules of WTO agreements are working together 
to achieve the main principles of WTO.

Morover, this paper justifies the use of textual information inside the textual description of the dispute and legal article by showing that simply using the co-citation pattern between articles of the WTO disputes can't qualitatively fit the $G^*$ (\textit{See} Figure \ref{fig:adj} and Figure \ref{}). 
Upon this necessity of using the textual information, this paper also justifies the use of neural network that is computationally intensive since it's generally known that proper design of neural network is able to effectively extract information from the textual content.

Finally, I will explain how this fitting of network of legal articles can contribute to the current study of international normative system.
