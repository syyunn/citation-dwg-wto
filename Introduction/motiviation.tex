As WTO sets its main principles to regulate the world trade system, 
such as \textit{Market Acceess} (across borders), 
\textit{Non-discrimination} (between members 
or between domestic products and imported products) 
and \textit{Transparency} (in publication and maintaining 
of each member's internal regulations), 
it's intellectually intriguing 
to understand how regulatory system of WTO DSB
is structured to achieve these main principles (\textit{See} Figure \ref{fig:market-aceess_directed}).
By understanding this structure, 
we can improve WTO system to serve its main prinicples more effectively 
and to adopt to constantly 
changing world trade circumstances
\citep{FREDEBEULKREIN1999625, shaffer_2004, 10.1093/jiel/jgm028}.

However, it is extremely difficult to 
understand how rules of the WTO agreements are
working together to achieve those main principles of WTO. 
This is because each citation is closely related 
to complex characteristics 
of each trade policy as exemplified in 
the above mentioned CDOSA case. 
Moreover, to understand interactions between multiple rules of WTO agreements properly, 
it requires one to generalize members' strategic 
citation patterns which are limited 
to each member's special interest 
rather than explaining the 
regulatory system of WTO DSB in general.


