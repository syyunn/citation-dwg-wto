As WTO sets multiple principles to regulate the world trade system, 
such as \textit{Market Acceess} (across borders), 
\textit{Non-discrimination} (between members 
or between domestic products and imported products) 
and \textit{Transparency} (in publication and maintaining 
of each member's internal regulations), 
it's intellectually intriguing 
to understand how regulatory system of WTO DSB
is structured to achieve these core principles.
By understanding this structure, 
we can improve WTO system more efficiently 
to adopt to constantly 
changing world trade circumstances
\citep{FREDEBEULKREIN1999625, shaffer_2004, 10.1093/jiel/jgm028}.
However, it is extremely difficult to 
understand how regulatory system of 
WTO DSB is organized to acheive 
those core principles. 
This is because each citation is closely related 
to multiple characteristics 
of each trade policy as exemplified in 
the above mentioned \textit{Byrd Amendment} case. 
Moreover, to understand the system properly, 
it requires to generalize strategic 
citation patterns which are limited 
to each member's special interest 
rather than explaining the 
regulatory system of WTO DSB in general.


