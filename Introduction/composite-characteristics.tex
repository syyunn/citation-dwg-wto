A lawsuit tends to cite multiple rules of the WTO agreement because a trade policy is usually pretty much complicated
and one simple rule can't cover the multiple characteristics of the trade policy that led to the dispute \citep{palmeter2004dispute}.
For example, the United States enacted \textit{Continued Dumping and Subsidy Act of 2000 (CDSOA)} that distributes
the collected anti-dumping duties to its affected domestic producers and this act was challenged with multiple rules of the WTO agreements in the \textit{US - Offset (Byrd Amendment)} case,
such as \textit{Anti-dumping} and \textit{Subsidy}, because
this distribution could constitute a illegal subsidy prohibited
in the rules of the WTO agreements in addition to the rules of the anti-dumping. %\citep{cdsoa}.
Moreover, members' citation of the rules of the WTO agreements becomes even more complicated because members cite the
rules of the WTO agreements strategically. For example,
members cite different rules of the WTO agreements to limit or to encourage
the third party participation because third party
participation can lead to early settlement of the dispute without continuous
legal battle and vice versa  \cite{who_gets}.
Things are getting more interesting
if we consider the fact WTO DSB defers to legal precedents.
Legal precedents refer to its own judicial decisions
and they provides authoritative reference
for deciding subsequent identical or similar cases.
Members try to reshape these legal precedents
in favor of their future interest rather than
simply using the WTO DSB to resolve their specific trade
dispute \citep{pelc}. For example,
members cite their
favorable previous cases more often in specific
issue areas where they face ligitagion more frequently with other members.
\citep{latent}.
 
 % \footnote{
% \blockquote{
%     The CDSOA is a new and complex measure, applied in a complex legal environment. In
% concluding that the CDSOA is in violation of the abovementioned provisions, we have been
% confronted by sensitive issues regarding the use of subsidies as trade remedies. If Members are of the
% view that subsidisation is a permitted response to unfair trade practices, we suggest that they clarify
% this matter through negotiation
% }
% }

