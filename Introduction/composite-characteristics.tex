This process requires enormous legal knowledge and resources because the legal system of WTO is highly complex.
This complexity has constrained many developing countries with limited legal knowledge and resources from fully utilizing the WTO DSB
\citep{busch_reinhardt_shaffer_2009, dev_busch, shaffer_2006}. 
 
To address this issue, I provide a novel method to summarize the network of WTO articles.
Currently, understanding of how articles of WTO agreements systematically interact with each other is exclusively shared among legal experts.
However, by developing the method that can quantitatively summarize the systematic interaction between articles of WTO,
we can lower the cost of understanding the legal system of WTO. This will help resolve the unbalanced legal capacity issue in WTO DSB.
 
To properly summarize the systematic interactions between articles of WTO agreements, I designed my method based on two following considerations.
First, since the legal system of WTO evolves from the way how real-world dispute interacts with the regulatory content of the article of WTO agreement,
I considered a way of utilizing two different types of textual data, factual description of the trade dispute and the content of each article of the WTO agreements.
Second, since members strategically cite rules of the WTO agreements to encourage
the third party participation \citep{who_gets} or to reshape the legal precedents\citep{pelc, latent},
I considered a way of generalizing these member-specific strategic citations.
 
Upon these two considerations, this paper uses deep learning.
Deep learning is empirically known as good at extracting information from the textual data.
In addition to it, deep learning also generalizes the patterns inside data.
Therefore, this paper designs a deep neural network that
processes two different types of textual data,
description of the dispute and each article content of the WTO agreements.
The design mimics the reasoning process of the legal experts,
where the experts read
the textual description of the dispute and imagine applicable legal articles of the WTO agreements according to its regulatory content.
 
To train this neural network, I collected textual description of trade dispute and articles of the WTO agreement cited for each dispute requested to the WTO DSB
from 1995 to 2018.
Using this collected data, I trained the neural network by enforcing the neural network to answer correctly
whether a given article of the WTO agreements
can be cited for the given textual description of
trade dispute.
After training, I fitted a network that summarizes the systematic interactions between articles of WTO agreements using \textit{Random Forests} \citep{rf, genie3}.
The network is fitted as to best explain the variance of each article's citabilities. Those citabilities are collected from the predictions of the trained deep neural network.
 
To verify the quality of the fitted network, I compared the fitted network with the jurisprudence of WTO DSB appearing in the Panel and Appellate Body reports.
Specifically, I found three major principles of WTO DSB, \textit{Market Access}, \textit{Reciprocity}, and \textit{Non-discrimination}, are clustered in the fitted network.
The systematic interactions between articles of WTO agreements are formed as how the Panel and Appellate Body explained in their judicial opinions.
As Panel and Appellate Body authoritatively constitute the jurisprudence of WTO DSB, one can conclude that the method qualitatively summarizes the systematic interactions of articles of WTO agreements.
 
% % == DEF EXAMPLE == 
\begin{figure}[t!]
  \captionsetup[subfigure]{justification=centering}
  \begin{subfigure}[b]{1\textwidth}
      \centering{
              \begin{tikzpicture}[
        >={Stealth[color=black]}
        ,shorten >=1pt,node distance=2cm
        ,on grid,initial/.style={}
        ,every label/.style={align=left}
        ]
        \linespread{2}
        \node[state, label=above:{National Treatment \\[1mm] \hspace{0mm} on Internal Taxation}] (T1) at (10, 0) {$v_j$ = III:2};

        \node[text width=5cm] at (8.5,0)
        {$\sum_{v_i\in V}{w_{ij}} = 1$};

        \draw[ultra thick, gray!50, dotted, ->] (13,1.8) arc (30:-170:3.5);
        \draw[ultra thick, gray!50, dotted, -] (13,1.8) arc (30:170:3.5);

        \node[state, label=right:{Interpretation of \\[1mm] \hspace{-1.5mm} Tariff Concession}] at ([shift=({0:5 cm})]T1) (T2) {$v_i$ = II:1(b)};
        \node[state, label=right:{General Elimination of \\[1mm] \hspace{-2mm} Quantitative Restrictions}] at ([shift=({-30:5 cm})]T1) (T3) {XI:1};
        \node[state, label=right:{Fair Adminstration of \\[1mm] \hspace{-1mm} Laws and Regulations}] at ([shift=({30:5 cm})]T1) (T4) {X:3(a)};
        
        \draw[red,arrows={[red]<-}, ultra thick] node[above, xshift=12.75cm] {$P(v_j|v_i) = 8\%$} (T1) -- (T2);

        \begin{scope}[every edge/.append style={<-}] % for directed edge, change "style={->, double=black, draw=white}]"
            \path
            % (T1) edge[red, ultra thick] node[above] {$0.092$} (T2)
            (T1) edge[double=black, draw=white] node[above, xshift=5pt] {$11\%$} (T3)
            (T1) edge[double=black, draw=white] node[above, yshift=5pt] {$9.7\%$} (T4);

            \path[->] (T1) edge[thin,<-,densely dotted] node[above, xshift=5pt] {$w_{i_{1}j}$} +(1.95,-3.3);
            \path[->] (T1) edge[thin,<-,densely dotted] node[above, xshift=10pt] {$w_{i_{2}j}$} +(0,-3.75);
            \path[->] (T1) edge[thin,<-,densely dotted] node[above, xshift=10pt] {$w_{i_{3}j}$} +(-1.95,-3.3);
        \end{scope}

    \end{tikzpicture}

% \begin{figure}[ht]
%     \centering
    
%     \caption{\textbf{Example of Network of Legal Articles of WTO agreements: }}
%     \label{fig:def-example}
% \end{figure}

      }
      \caption{\textbf{Illustration of $P(v_j \mid v_i)$ where the target article $v_j=$ Article III:2}}
      \label{subfig:a:art2b}
  \end{subfigure}
  \vfill
  \begin{subfigure}[b]{1\textwidth}
      \centering{
          \begin{displayquote}[][]
    \begin{center}
    \end{center}
  
    \begin{displayquote}[][]
    ``The dictionary definition of the noun `excess' is `[t]he amount by which one number
        or quantity exceeds another'. More specifically, `in excess of' means `more than'. Thus,
        as a textual matter, a particular number or quantity is `in excess of' another number
        or quantity if it is greater, regardless of the extent to which it is greater. 
      \textbf{\textit{Looking at the context of Article II:1(b), first sentence, we note that Article III:2, first
      sentence, of the GATT 1994 is cast in very similar terms and in fact uses the phrase
      `in excess of'}}:\\
        \begin{displayquote}
        The products of the territory of any contracting party imported into the
      territory of any other contracting party shall not be subject … to internal
      taxes or other internal charges of any kind in excess of those applied … to
      like domestic products \ldots
        \end{displayquote}   
    \end{displayquote}  
  \end{displayquote}

      }
      \centering
      \caption{\textbf{Jurisprudence of Panel in \textit{Russia – Tariff Treatment} case:} Panel explains that the meaning of the term \textit{`in excess of'} in Article II:1(b) clarifies the meaning of the same phrase in Article III:2.}
      \label{subfig:a:condprob}
  \end{subfigure}
  \caption{\textbf{Modeling of the Network of Articles of WTO agreements}}
  \label{fig:def-example}
\end{figure}

\section{Modeling and Formal Definitions}
\subsection{Network of Articles of the WTO agreements \label{subsec:def}}

I define the network of articles of WTO agreements as directed weighted graph $G=(V, \vec{E}, W)$ which is comprised of 
vertex set $V$, set of directed edges $\vec{E}$, and edge weight matrix $W$. I define each legal article of WTO agreement as a vertex, thus
$V=\{v \mid v\text{ is a legal article of WTO agreement\}}$. Then I define all ordered pairs of vertices as a set of directed edges $\vec{E}$, thus
$\vec{E} = \{(v_i, v_j) \mid (v_i, v_j)\in V \times V)\}$. 
Finally, I define the edge weight matrix $W=(w(v_i, v_j)) \in [0,1]^{|V| \times |V|}$ where all incoming edge weights sum up to $1$ for all given target vertex $v_j$, thus
$\sum_{v_i\in V}{w(v_i, v_j)}$ = 1. $w$ denotes a map that assigns a weight for each ordered pair of vertices, thus $w : V \times V \to [0,1]$. 
I always assign weight $0$ for the directed edge comprised of the same vertex, thus  $w(v_i, v_i) = 0 \:\: \forall{v_i \in V}$.
For convenience, I define $w_{ij} = w(v_i, v_j)$.


\subsection{Modeling Interaction between Articles of WTO agreements as Conditional Probability}
I interpret every directed edge weight $w(v_i, v_j)$ as the conditional probability $P(v_j|v_i) \in [0,1]$, where the probability represents how probably the source node $v_i$ clarifies the interpretation of the target node $v_j$ comapred to all other source nodes $v\in V \setminus \{v_i, v_{j}\}$. 
The articles of WTO agreements interdependently constitute the legal context to clarify the interpretation of other articles as shown in the the Panel report of \textit{Russia-Tariff Treatment} case, as excerpted in Figure \ref{subfig:a:condprob}. In \textit{Russia-Tariff Treatment} case, the Panel explained that Article II:1(b) clarifies the meaning of the same phrase \textit{`in excess of'} in Article III:2.
By modeling this clarification relationship as the directed edge weight $w_{ij}$, I let the edge weight $w_{ij}$ represent the realtive importance of a source article $v_i$ clarifying the interpretation of the target article $v_j$. I illustrated this relationship in Figure \ref{subfig:a:art2b}.

\subsection{Methodological Objective: Finding $G^*$}
I aim to find $G^* = (V, \vec{E}, W^*)$ where the $W^*$ closely reflect the clarification relationship between articles of WTO agreements as explained by the authoritative judicial bodies of the WTO DSB, Panel and Appellate Body. To find $W^*$, this paper collected the past 20 years of legal dispute data in WTO DSB. The types and composition of the data collected will be explained in Section \ref{sec:data}.
Then I design a deep neural network to encode the pattern of interactions of the articles of WTO agreements found in the data. Justification of using deep learning, design and training of deep neural network, and fitting process of $W^*$ using \textit{Random Forest} will be explained in Section \ref{sec:methods}.
Finally in Section \ref{ef}, I will verify the quality of the fitted $G^*$ by comparing the systematic interaction between articles of WTO agreements found in $G^*$ with the corresponding jurisprudence of the Panel and Appellate Body.

% network $G = (V, \vec{E}, W)$ collect all the relative importance of each article compared to other articles to 

% every directed edge weight $w(v_i, v_j)$ 

% model this clarification relationship as the directed edge weight $w(v_i, v_j)$.

% Above subfigure (a) represents how jurisprudence of \textit{Panel} stated in (b) is represented as an edge weight where the source node Article II:1(b) constitutes the legal context of the target node Article III:2 regarding how to interpret its term \textit{`in excess of'} with the $8\%$ of importance compared to other possible source articles.}

% The system has become more complex as the number of cases requested to WTO DSB increases and developing countries can't resolve their dissatisfaction over the trade relationship in WTO DSB.

% Moreover, I aim to summarize the network of WTO articles as a whole not by parts.
% There already exists numerous literatures that studied the relationship between articles of WTO agreements \citep{chadXXIII, charnovitz, Trachtman, who_gets}.
% However, previous literatures have been limited to studying the interconnections between relatively small number of articles, which are mostly less than 10. % and has not been conducted in a level of entire WTO agreements. As my method captures the entire systematic correlation between the articles of WTO agreements, it becomes 
% By pursuing a holistic approach, I intend to enhance the practicalness of the method. The legal system of WTO always works as whole and the entire system provide legal contexts to its sub-systems \citep{system_as_a_whole}.

% In other words, this method effectively materializes the shared understanding of legal experts thus can lower the cost to build the same amount of legal capacity to understand the WTO DSB.
% Moreover, rather than keep relying on previous approach that provides legal advice to developing countries that does not create a shared understanding over the system between developing and developed countries,
% if we shift our focus on how to materialize the current shape of the system, WTO
% will become more effective as members being able to discuss their trade issues upon the measurable ground of shared understanding about how WTO works.




% way of the fitted network $G^*$ explaining how articles of WTO agreements achieve some main principles of WTO, such as \textit{Market Access}, \textit{Reciprocity} and \textit{Non-discrimination}



% fitted a set of directed edge weights $W^*$ that
% best explains the variance of each article's citabilities that are predicted by the trained deep neural network using a machine learning technique \textit{Random Forests} \citep{rf, genie3}.
 
% After fitting $W^*$, to check whether this fitted network of articles of the WTO agreements\\  $G^*$ = ($V$, $E$, $W^*$) maps the jurisprudences of WTO DSB properly, this paper
% compares the way of the fitted network $G^*$ explaining how articles of WTO agreements achieve some main principles of WTO, such as \textit{Market Access}, \textit{Reciprocity} and \textit{Non-discrimination}
% with the jurisprudences of \textit{Panel} and \textit{Appellate Body}. This comparison reveals that the fitted network $G^*$ captures the interactions between the articles of WTO agreements
% closely to the jurisprudences of \textit{Panel} and \textit{the Appellate Body}. We can infer from this similarity that the fitted network $G^*$ closely maps the jurisprudences of WTO DSB. This is because those two judicial bodies
% authoritatively constitute the jurisprudences over how rules of WTO agreements are working together
% to achieve those main principles.
 
% Finally, upon this similarity, this paper offers this methodology as
% an alternative solution to the widening gap of legal capacity between developing and developed countries in WTO DSB.
% Since this method effectively materializes the shared understanding of legal experts and reveals important interactions between articles inside the system of WTO DSB,
% it can lower the cost to build the same amount of legal capacity to understand the WTO DSB.
% Moreover, rather than keep relying on previous approach that provides legal advice to developing countries that does not create a shared understanding over the system between developing and developed countries,
% if we shift our focus on how to materialize the current shape of the system, WTO
% will become more effective as members being able to discuss their trade issues upon the measurable ground of shared understanding about how WTO works.
 



% Therefore, this paper designs a deep neural network (\textit{See} Figure \ref{fig:design-of-nn} and \ref{fig:dnn-flow}) that
% processes two different types of textual information.
% One is textual description of the dispute (\textit{See} an example at \hyperref[sub:factual-aspect-example]{Appendix A.1}) and
% the other one is the text of a legal article of the WTO agreements (\textit{See} an example at Figure \ref{fig:gatt_art1}).
% This design is improvised to mimic
% the reasoning process of WTO legal practitioners
% where the legal practitioners read
% the textual description of
% factual circumstances of the dispute and imagine applicable regulatory contents of
% the legal articles while he/she reads that factual description of the case (\textit{See} Figure \ref{fig:viz:how-member-cites-citable}, \ref{fig:viz:how-member-cites-non-citable} and \ref{fig:design-of-nn}).


% The legal system of WTO is understood as a complex network of articles of WTO agreements.
% The articles of WTO agreements interact with each other to constitute specific norms and regulates specific trade issues.
% There exists a numerous literatures that studies the relationship between articles of WTO agreements \citep{chadXXIII, charnovitz, Trachtman, who_gets}.
% However, previous literatures have been limited to study the interconnection between relatively small number of articles, which are mostly less than ten. % and has not been conducted in a level of entire WTO agreements.
% Since the legal system of WTO comprises 


% To address this issue, I provide a novel method to summarize the network of WTO articles as a whole.




% aims to find out the network of legal articles of WTO agreements as a whole.

% This kind of research that studies the relationship between articles of WTO agreements has been actively pursued in numerous literatures \citep{chadXXIII, charnovitz, Trachtman, who_gets}, however, those efforts has been limited to


% The entire map of this network between articles of WTO agreements has been exclusively shared among a group of legal experts of the WTO agreements.



% network of interconnection between articles of WTO agreements is exclusively shared among a group of legal experts of the WTO agreements. 
% This exclusiveness becomes more severe as the number of cases requested to WTO DSB increases and this has led to a widening gap of legal capacities between developing and developed countries.
% This gap now inhibits the effectiveness of the WTO because developing countries are excluded from the WTO DSB to resolve their dissatisfaction over the trade relationship with other members.
% The legal system of WTO is understood as a complex network of articles of WTO agreements.




% This process requires enormous legal knowledge and resources.


% A trade dispute tend to involve complex issue structure.


% because it requires many legal experts to structure a legal argument with a full understanding of the WTO legal system.


% and 

% The legal system of WTO is understood as a complex network of articles of WTO agreements.
% Each part of the network constitutes specific norms and regulates specific trade issues.




% where each part of network handles each 


% between articles are constituting specific norms and regulating specific trade issues.


% Countries usually cite multiple articles to claim their impaired benefits in WTO DSB.

% and the legal system of WTO DSB has evolved into complex network of articles where the interconnectedness between articles are constituting specific norms and regulating specific trade issues.

% This process requires enormous legal knowledge and resources because a trade dispute tend to involve several interconnected trade issues.

% to handle those trade disputes.

% analyzing the network of articles of the WTO agreements provides us a clear view on how WTO DSB constitutes specific norms and regulates specific trade issues.


% Therefore, countries tend to cite multiple articles of WTO agreements to claim their impaired benefits in WTO DSB.

% % Citation of articles of the WTO agreements are evidence of how WTO rules are applied in the real world.


% Therefore, a lawsuit tends to cite multiple rules of the WTO agreement because one simple rule can't cover the complex characteristics of the trade policy that led to the dispute \citep{palmeter2004dispute}.
% For example, the United States enacted \textit{Continued Dumping and Subsidy Act of 2000} (CDSOA) that distributes
% the collected anti-dumping duties to its affected domestic producers.
% This act was challenged by other members with several rules of
% the WTO agreements such as \textit{rules of anti-dumping} and \textit{rules of subsidy}. This was because
% this distribution could constitute an illegal subsidy and illegal anti-dumping duty at the same time as stated in the Panel report of the \textit{US - Offset (CDSOA)} case%(\textit{See} Figure \ref{fig:complex-measure}).

% \blockquote{
%   8.1 In the light of our findings, we conclude that \textbf{the CDSOA is inconsistent with AD (Anti-dumping)
%   Articles 5.4, 18.1 and 18.4, SCM (Subsidy and Countervailing Measure) Articles 11.4, 32.1 and 32.5,} Articles VI:2 and VI:3 of the GATT
%   1994, and Article XVI:4 of the WTO Agreement. \ldots\\
%   8.3 \textbf{The CDSOA is a new and complex measure, applied in a complex legal environment}. In
%   concluding that the CDSOA is in violation of the above mentioned provisions, we have been
%   confronted by sensitive issues regarding the use of subsidies as trade remedies.
%   this matter through negotiation.
% }

% % Citation of articles of the WTO agreements are evidence of how WTO rules are applied in the real world.
% Since multiple articles cooperate to deal with the real world dispute, analyzing the network of articles of the WTO agreements provides us a clear view on how WTO DSB constitutes specific norms and regulates specific trade issues.
% % Principle of \textit{Market Access} which is one of the most important principles in WTO is achieved through a cooperation of multiple articles of the WTO agreements %(Figure \ref{fig:market-aceess_directed}) 
% % and those cooperation is realized through the act of citation by members.
% % Therefore, analyzing the network of articles of the WTO agreements provides us a clear view on how WTO DSB constitutes specific norms and regulates specific trade issues.
% This kind of research that studies the relationship between articles of WTO agreements has been actively pursued in numerous literatures \citep{chadXXIII, charnovitz, Trachtman, who_gets}, however, those efforts has been limited to
% study interconnectedness between relatively small numbers (less than 10) of articles. % and has not been conducted in a level of entire WTO agreements.

% Therefore, this research aims to find out the network of legal articles of WTO agreements as a whole.
% Currently, the entire map of interconnection between articles of WTO agreements is exclusively shared among a group of legal experts of the WTO agreements.
% This exclusiveness becomes more severe as the number of cases requested to WTO DSB increases and this has led to a widening gap of legal capacity between developing and developed countries.
% This gap now inhibits the effectiveness of the WTO because developing countries are excluded from the WTO DSB to resolve their dissatisfaction over the trade relationship with other members.
% % at this moment and members are discussing possible solutions under the agenda of \textit{WTO reform}
% % but proposed solutions are mainly about unilateral support of legal resources from the developed countries to developing countries and this solution has been ineffectiveness reflecting on the similar efforts during the past decades of WTO DSB.
% % such as principle of \textit{Market Acceess} (across borders),
% % \textit{Non-discrimination} (between members
% % or between domestic products and imported products)
% % and \textit{Transparency} (in publication and maintaining
% % of each member's internal regulations),
% % it's intellectually intriguing
% % to understand how regulatory system of WTO DSB
% % is structured to achieve these main principles (\textit{See} Figure \ref{fig:market-aceess_directed}).
% % By understanding this structure,
% % we can improve WTO system to serve its main prinicples more effectively
% % and to adopt to constantly
% % changing world trade circumstances
% % \citep{FREDEBEULKREIN1999625, shaffer_2004, 10.1093/jiel/jgm028}.
% % Therefore, this paper emphasizes the importance of revolutionizing the way of studying network of articles of WTO agreements 

% Therefore, this paper provides a novel method to summarize the network of articles of WTO agreements. 
% % This network materializes the relationship between the articles of the WTO agreements as a .
% This paper maps
% the jurisprudence of WTO DSB
% as a network of legal articles
% as formally defined as

% in Figure \ref{fig:def} and illustrated in Figure \ref{fig:def-example}. This is because the rules of the WTO agreements
% explicitly requires \textit{Panel} or \textit{Appellate Body} to address
% relevant articles together when they construct its jurisprudence related to the meaning, scope and interpretation of any legal text in the WTO agreements as excerpted in Figure \ref{fig:art7dsu}.
% Upon this requirement, judicial bodies cite
% multiple articles together
% to identify the complex legal identity of a trade policy at issue as clearly opinionated in Figure \ref{fig:complex-measure}.
% In addition to it, judicial bodies cite multiple articles together
% to guide an way of interpretation of the rules of the WTO agreements (\textit{See} Figure \ref{subfig:a:condprob}).
% % \begin{figure}[h]
% %   \begin{quote}
% %       8.1 In the light of our findings, we conclude that \textbf{the CDSOA is inconsistent with AD (Anti-dumping)
% %       Articles 5.4, 18.1 and 18.4, SCM (Subsidy and Countervailing Measure) Articles 11.4, 32.1 and 32.5,} Articles VI:2 and VI:3 of the GATT
% %       1994, and Article XVI:4 of the WTO Agreement. \ldots
% %   \end{quote}
% %   % \begin{quote}
% %   %     \centering{\ldots}
% %   % \end{quote}
% %   \begin{quote}
% %       8.3 \textbf{The CDSOA is a new and complex measure, applied in a complex legal environment}. In
% %       concluding that the CDSOA is in violation of the above mentioned provisions, we have been
% %       confronted by sensitive issues regarding the use of subsidies as trade remedies.
% %       this matter through negotiation.
% %   \end{quote}
% %   \caption{\textbf{Panel's Judicial Opinion On the \textit{US - Offset (Byrd Amendment; CDSOA)} case:} Panel explicitly expresses the complexity of the trade policy (CDSOA) at issue and cites the rules of anti-dumping (AD) and subsidy (SCM) at the same time to cover its complex characteristics.}
% %   \label{fig:complex-measure}
% % \end{figure}
% To develop a proper method that can find a set of directed edge weights $W$ defined in Figure \ref{fig:def}
% as close to a shared understanding of legal experts, this paper points out two main considerations.
% First, one need to use information inside a textual description of factual circumstances of the legal dispute and the regulatory contents inside the text of each article of the WTO agreements.
% Second, one need to generalize the members' strategic citation pattern that is limited to a member-specific political interest.
% For example, members strategically cite different rules of the WTO agreements to limit or to encourage
% the third party participation. Since the third party participation
% can lead to early settlement of the dispute without continuous
% legal battle, members cite differently according to their intention to
% settle the case earlier out of court \citep{who_gets}. Moreover, members cite articles strategically trying to reshape the legal precedents of WTO DSB
% in favor of their future interest \citep{pelc, latent}.
% % For example,
% % members tend to cite their
% % favorable previous cases more often in issue areas where they face ligitagion more frequently with other members
% % \citep{latent}.
% Upon these two considerations, this paper selects the deep neural network as a technical solution. % to effectively extract information from the text and to generalize the member specific strategic citation patterns.
% This is because a deep neural network is empirically known as good at extracting information from text and generalizing the patterns inside data.
% Therefore, this paper designs a deep neural network (\textit{See} Figure \ref{fig:design-of-nn} and \ref{fig:dnn-flow}) that
% processes two different types of textual information.
% One is textual description of the dispute (\textit{See} an example at \hyperref[sub:factual-aspect-example]{Appendix A.1}) and
% the other one is the text of a legal article of the WTO agreements (\textit{See} an example at Figure \ref{fig:gatt_art1}).
% This design is improvised to mimic
% the reasoning process of WTO legal practitioners
% where the legal practitioners read
% the textual description of
% factual circumstances of the dispute and imagine applicable regulatory contents of
% the legal articles while he/she reads that factual description of the case (\textit{See} Figure \ref{fig:viz:how-member-cites-citable}, \ref{fig:viz:how-member-cites-non-citable} and \ref{fig:design-of-nn}).
% % As WTO sets its main principles to regulate the world trade system in general
% % such as principle of \textit{Market Access} (across borders),
% % \textit{Non-discrimination} (between members
% % or between domestic products and imported products)
% % and \textit{Transparency} (in publication and maintaining
% % of each member's internal regulations),
% % it's intellectually intriguing
% % to understand how regulatory system of WTO DSB
% % is structured to achieve these main principles (\textit{See} Figure \ref{fig:market-aceess_directed}).
% % By understanding this structure,
% % we can improve WTO system to serve its main principles more effectively
% % and to adopt to constantly
% % changing world trade circumstances
% % \citep{FREDEBEULKREIN1999625, shaffer_2004, 10.1093/jiel/jgm028}.
% % Moreover, members' citation of articles of the WTO agreements also gets complicated
% % if we consider the fact that \textit{Panel} or \textit{Appellate Body} defers to the legal precedents of WTO DSB.
% % Legal precedents refer to \textit{Panel} or \textit{Appellate Body}'s judicial decisions
% % and these legal precedents provide authoritative reference
% % for deciding subsequent cases in WTO DSB.
% % Members try to reshape these legal precedents
% % in favor of their future interest more than
% % simply using the WTO DSB to resolve their trade
% % dispute \citep{pelc}. For example,
% % members tend to cite their
% % favorable previous cases more often in issue areas where they face ligitagion more frequently with other members
% % \citep{latent}.
% % as close to
% % a shared understanding of legal experts of the WTO agreements, one needs to understand how members cites articles of the WTO agreements in WTO DSB.
% % However, Citation of the rules of the WTO agreements tends to get more complicated because members cite the
% % rules of the WTO agreements strategically.
% % \footnote{
% % \blockquote{
% %     The CDSOA is a new and complex measure, applied in a complex legal environment. In
% % concluding that the CDSOA is in violation of the above mentioned provisions, we have been
% % confronted by sensitive issues regarding the use of subsidies as trade remedies. If Members are of the
% % view that subsidisation is a permitted response to unfair trade practices, we suggest that they clarify
% % this matter through negotiation
% % }
% % }
 
 
 

