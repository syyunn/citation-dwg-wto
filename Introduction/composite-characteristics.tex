A lawsuit tends to cite multiple rules of the WTO agreement because one simple rule can't cover the complex characteristics of the trade policy that led to the dispute \citep{palmeter2004dispute}.
For example, the United States enacted \textit{Continued Dumping and Subsidy Act of 2000} (CDSOA) that distributes
the collected anti-dumping duties to its affected domestic producers.
This act was challenged by other members with multiple rules of
the WTO agreements such as \textit{rules of anti-dumping} and \textit{rules of subsidy} because
this distribution could constitute an illegal subsidy and anti-dumping duty at the same time (\textit{See} Figure \ref{fig:complex-measure}).

Citation of articles of the WTO agreements are evidence of how WTO rules are applied in the real world. 
For example, principle of \textit{Market Acceess} which is one of the most important principles in WTO, is achieved through the cooperation of multile articles of the WTO agreements (\textit{See} Figure \ref{fig:market-aceess_directed}) where the cooperation is realized thorugh the act of citation.
Therefore, analyzing the network of articles of the WTO agreements provides us a clearer view on how WTO DSB constitutes specific norm or regulates specific trade issues. 
This kind of study has been actively pursued by a group of researchers in numerous literatures \citep{chadXXIII, charnovitz, Trachtman, who_gets}, however, those efforts has been limited to 
study interconnectedness between relatively small number (less than 10) of articles of the WTO agreements and has not been conducted in a level of entire WTO agreements.
There exists an understanding about how rules of the WTO agreements are interconnected as a whole, however, this entire map is exclusively shared among a group of legal experts and researchers of the WTO agreements.
This exclusiveness becomes more severe as the number of requested cases increases to WTO DSB and this has led to a widening gap of legal capcity between developing and developed countries in WTO. 
This gap now inhibits the effectiveness of the WTO because developing countries are excluded from the WTO DSB to resolve their dissatisfaction over the trade relationship with other members.

% at this moment and members are discussing possible solutions under the agenda of \textit{WTO reform} 
% but proposed solutions are mainly about unilateral support of legal resources from the developed countries to developing countries and this solution has been ineffectiveness reflecting on the similar efforts during the past decades of WTO DSB.


% such as principle of \textit{Market Acceess} (across borders), 
% \textit{Non-discrimination} (between members 
% or between domestic products and imported products) 
% and \textit{Transparency} (in publication and maintaining 
% of each member's internal regulations), 
% it's intellectually intriguing 
% to understand how regulatory system of WTO DSB
% is structured to achieve these main principles (\textit{See} Figure \ref{fig:market-aceess_directed}).
% By understanding this structure, 
% we can improve WTO system to serve its main prinicples more effectively 
% and to adopt to constantly
% changing world trade circumstances
% \citep{FREDEBEULKREIN1999625, shaffer_2004, 10.1093/jiel/jgm028}.



Therefore, this paper addresses importane of revolutionizing the way of studying network of articles of WTO agreements and propose a new methodology to
materialize the relationship between the articles of the WTO agreements as a whole. 
For this purpose, this paper maps
the regulatory system of WTO DSB
as a network of legal articles
of the WTO agreements as formally defined in Figure \ref{fig:def} and illustrated in Figure \ref{fig:def-example}. This is because the rules of the WTO agreements
explicitly requires \textit{Panel} or \textit{Appellate Body} to address
relevant articles together when they construct its jurisprudence related to the meaning, scope and interpretation of any legal text in the WTO agreements as excerpted in Figure \ref{fig:art7dsu}.
Upon this requirement, judicial bodies cite
multiple articles together
to identify the complex legal identity of the trade policy as exemplified in Figure \ref{fig:complex-measure}.
In addition to it, judicial bodies cite multiple articles together
to guide an way of interpretation of the rules of the WTO agreements (\textit{See} Figure \ref{subfig:a:condprob})




\begin{figure}[h]
    \begin{quote}
        8.1 In the light of our findings, we conclude that \textbf{the CDSOA is inconsistent with AD (Anti-dumping)
        Articles 5.4, 18.1 and 18.4, SCM (Subsidy and Countervailing Measure)} Articles 11.4, 32.1 and 32.5, Articles VI:2 and VI:3 of the GATT
        1994, and Article XVI:4 of the WTO Agreement. \ldots
    \end{quote} 
    % \begin{quote}
    %     \centering{\ldots}
    % \end{quote}
    \begin{quote}
        8.3 \textbf{The CDSOA is a new and complex measure, applied in a complex legal environment}. In
        concluding that the CDSOA is in violation of the above mentioned provisions, we have been
        confronted by sensitive issues regarding the use of subsidies as trade remedies.
        this matter through negotiation.
    \end{quote} 
    \caption{\textbf{Panel's Judicial Opinion On the \textit{US - Offset (Byrd Amendment; CDSOA)} case:} Panel explicitly expresses the complexity of the trade policy (CDSOA) at issue and cites the rules of anti-dumpig (AD) and subsidy (SCM) at the same time to cover its complex characteristics.}
    \label{fig:complex-measure}
\end{figure}

To develop a proper method that can find a set of directed edge weights $W$ defined in Figure \ref{fig:def}
as close to a shared understanding of legal experts, this paper points out two main considerations. 
First, one need to use information inside a textual description of factual circumnstances of the dispute and the reuglatory contents described in article of the WTO agreements.
Second, one need to generalize the members' strategic citation pattern that is limited to a member's specific political interest.
For example, members strategically cite different rules of the WTO agreements to limit or to encourage
the third party participation. Since the third party participation
can lead to early settlement of the dispute without continuous
legal battle, members cite differently according to their intention to
settle the case earlier out of court or vice versa \citep{who_gets}. Moeroever, members cite articles strategically trying to reshape the legal precedents of WTO DSB
in favor of their future interest \citep{pelc, latent}. 
% For example,
% members tend to cite their
% favorable previous cases more often in issue areas where they face ligitagion more frequently with other members
% \citep{latent}.

Upon these two considerations, this paper adopts the deep neural network as a technical solution. % to effectively extract information from the text and to generalize the member specific startegic citation patterns.
This is because a deep neural network is generally known as good at effectively extracting information from text data and generalizing the patterns inside data. 
Therefore, this paper designs a deep neural network (Figure \ref{fig:traditional-convolutional-network}) that
processes two different types of textual information.
One is textual description of the dispute (\textit{See} an example at \hyperref[sub:factual-aspect-example]{Appendix A.1}) and
the other one is the text of a legal article of the WTO agreements (\textit{See} an example at Figure \ref{fig:gatt_art1}).
This design is improvised to mimic
the reasoning process of WTO legal practitioners
where the legal practitioners read
the textual description of
factual circumstances of the dispute and imagine applicable regulatory contents of
the legal articles while he/she reads the factual description (\textit{See} \ref{fig:viz:how-member-cites-citable}, \ref{fig:viz:how-member-cites-non-citable} and \ref{fig:design-of-nn}).

% As WTO sets its main principles to regulate the world trade system in general
% such as principle of \textit{Market Acceess} (across borders), 
% \textit{Non-discrimination} (between members 
% or between domestic products and imported products) 
% and \textit{Transparency} (in publication and maintaining 
% of each member's internal regulations), 
% it's intellectually intriguing 
% to understand how regulatory system of WTO DSB
% is structured to achieve these main principles (\textit{See} Figure \ref{fig:market-aceess_directed}).
% By understanding this structure, 
% we can improve WTO system to serve its main prinicples more effectively 
% and to adopt to constantly
% changing world trade circumstances
% \citep{FREDEBEULKREIN1999625, shaffer_2004, 10.1093/jiel/jgm028}.




% Moeroever, members' citation of articles of the WTO agreements also gets complicated
% if we consider the fact that \textit{Panel} or \textit{Appellate Body} defers to the legal precedents of WTO DSB.
% Legal precedents refer to \textit{Panel} or \textit{Appellate Body}'s judicial decisions
% and these legal precedents provide authoritative reference
% for deciding subsequent cases in WTO DSB.
% Members try to reshape these legal precedents
% in favor of their future interest more than
% simply using the WTO DSB to resolve their trade
% dispute \citep{pelc}. For example,
% members tend to cite their
% favorable previous cases more often in issue areas where they face ligitagion more frequently with other members
% \citep{latent}.

% as close to
% a shared understanding of legal experts of the WTO agreements, one needs to understand how members cites articles of the WTO agreements in WTO DSB.



% However, Citation of the rules of the WTO agreements tends to get more complicated because members cite the
% rules of the WTO agreements strategically. 
% \footnote{
% \blockquote{
%     The CDSOA is a new and complex measure, applied in a complex legal environment. In
% concluding that the CDSOA is in violation of the abovementioned provisions, we have been
% confronted by sensitive issues regarding the use of subsidies as trade remedies. If Members are of the
% view that subsidisation is a permitted response to unfair trade practices, we suggest that they clarify
% this matter through negotiation
% }
% }

