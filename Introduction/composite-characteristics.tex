A lawsuit tends to cite multiple rules of the WTO agreement because one simple rule can't cover the complex characteristics of the trade policy that led to the dispute \citep{palmeter2004dispute}.
For example, the United States enacted \textit{Continued Dumping and Subsidy Act of 2000} (CDSOA) that distributes
the collected anti-dumping duties to its affected domestic producers.
This act was challenged by other members with multiple rules of
the WTO agreements such as rules of \textit{Anti-dumping} and rules of \textit{Subsidy} because
this distribution could constitute an illegal subsidy and anti-dumping duty at the same time (\textit{See} Figure \ref{fig:complex-measure}).

Citation of the WTO agreements are evidence of how WTO rules are applied in the real world. Therefore, analyzing how these rules are interconnected provide us with a clearer view how WTO DSB constitutes specific norm or regulates specific trade issues. 
This kind of efforts to study the interconnectedness between rules of WTO agreements has been actively pursued by a group of researchers, such as Chad P. Bown, Pelc, etc., however, those efforts has been limited to 
study interconnectedness between 2-3 articles of the WTO agreements and has not been conducted in a level of the entire WTO agreements as a whole.


% However, citation of the rules of the WTO agreeements are complex because of strategic 




\begin{figure}[h]
    \begin{quote}
        8.1 In the light of our findings, we conclude that \textbf{the CDSOA is inconsistent with AD (Anti-dumping)
        Articles 5.4, 18.1 and 18.4, SCM (Subsidy and Countervailing Measure)} Articles 11.4, 32.1 and 32.5, Articles VI:2 and VI:3 of the GATT
        1994, and Article XVI:4 of the WTO Agreement.
    \end{quote} 
    \begin{quote}
        \centering{\ldots}
    \end{quote}
    \begin{quote}
        8.3 \textbf{The CDSOA is a new and complex measure, applied in a complex legal environment}. In
        concluding that the CDSOA is in violation of the above mentioned provisions, we have been
        confronted by sensitive issues regarding the use of subsidies as trade remedies.
        this matter through negotiation.
    \end{quote} 
    \caption{\textbf{Panel's Judicial Opinion On the \textit{US - Offset (Byrd Amendment; CDSOA)} case:} Panel explicitly expresses the complexity of the trade policy (CDSOA) at issue and cites the rules of anti-dumpig (AD) and subsidy (SCM) at the same time to cover its complex characteristics.}
    \label{fig:complex-measure}
\end{figure}

% prohibited
% in the rules of the WTO agreements in addition to  %\citep{cdsoa}.

Citation of the rules of the WTO agreements tends to get more complicated because members cite the
rules of the WTO agreements strategically. For example,
members cite different rules of the WTO agreements to limit or to encourage
the third party participation to the case. Since the third party participation
can lead to early settlement of the dispute without continuous
legal battle, members strategically cite different rules of WTO agreements according to their intention to
settle the case earlier out of court or vice versa.
\citep{who_gets}.

In addition to it, members' citation of articles of the WTO agreements also gets complicated
if we consider the fact that \textit{Panel} or \textit{Appellate Body} defers to the legal precedents of WTO DSB.
Legal precedents refer to \textit{Panel} or \textit{Appellate Body}'s judicial decisions
and these legal precedents provide authoritative reference
for deciding subsequent cases in WTO DSB.
Members try to reshape these legal precedents
in favor of their future interest more than
simply using the WTO DSB to resolve their trade
dispute \citep{pelc}. For example,
members tend to cite their
favorable previous cases more often in issue areas where they face ligitagion more frequently with other members
\citep{latent}.

% \footnote{
% \blockquote{
%     The CDSOA is a new and complex measure, applied in a complex legal environment. In
% concluding that the CDSOA is in violation of the abovementioned provisions, we have been
% confronted by sensitive issues regarding the use of subsidies as trade remedies. If Members are of the
% view that subsidisation is a permitted response to unfair trade practices, we suggest that they clarify
% this matter through negotiation
% }
% }

