To address this issue,
this paper maps
the regulatory system of WTO DSB
as a network of legal articles
of the WTO agreements as formally defined in Figure \ref{fig:def} and exemplified in the Figure\ref{fig:market-aceess_directed}. This is because rules of the WTO agreements
% \footnote{\textit{See} Article 7 in Dispute Settlement Rules: Understanding on Rules and Procedures Governing the Settlement of Disputes, Marrakesh Agreement Establishing the World Trade Organization, Annex 2, 1869 U.N.T.S. 401, 33 I.L.M. 1226 (1994)} 
explicitly requires judicial bodies to address
relevant articles together to construct its jurisprudence as shown in Figure \ref{fig:art7dsu}.
Upon this requirement, judicial bodies refer to
multiple articles of the WTO agreements together
to identify the complex legal identity of the trade policy that led to the dispute.
In addition to it, judicial bodies cite multiple articles together
to provide an authoritative interpretation of the rules of the WTO agreements
\citep{oesch2003standards}.

\begin{figure}[ht]
    \[\text{Network of legal articles of WTO agreements is defined as}\] %\textit{directed weighted graph}}  G = (V, E, w) \]
    \[ \textit{directed weighted graph }G = (V, E, w) \]
    \[\text{ where } V = \{v \mid v\text{ is a legal article of WTO agreement}\}  \text{ , } \]
    \[\vec{E} = \{(v_i, v_j) \mid (v_i, v_j)\in V \times V)\} \] %\text{ and } \]
    \[w : V \times V \to \Bbb R_{+} \text{ } s.t. \sum_{v_j\in V}{w(v_i, v_j)} = 1 \text{ } \forall v_i \in V \text{ and }\]
    \[\text{simply define } w_{ij} := w(v_i, v_j)\]
    % \[\text{(}G\text{ is simply called \textit{directed weighted graph)}}\]
    \caption{\textbf{Formal Definition of Network of Legal Articles of WTO agreements: }
        We define network of legal articles of WTO agreements
        as a directed weighted graph where the sum of all weights coming out of a node sum up to 1 as illustrated in Figure \ref{fig:def-example}}
    \label{fig:def}
\end{figure}

\begin{figure}[ht]

    \centering
    \begin{tikzpicture}[
        >={Stealth[color=black]}
        ,shorten >=1pt,node distance=2cm
        ,on grid,initial/.style={}
        ,every label/.style={align=left}
        ]
        \linespread{2}
        \node[state, label=above:{Interpretation of \\[1mm] \hspace{-2mm} Tariff Concession}] (T1) at (10, 0) {II:1(b)};

        \node[text width=5cm] at (8.5,0) 
        {$\sum_{v_j\in V}{w_{ij}} = 1$};
        
        \draw[ultra thick, gray!50, dotted, ->] (13,1.8) arc (30:-170:3.5);
        \draw[ultra thick, gray!50, dotted, -] (13,1.8) arc (30:170:3.5);


        \node[state, label=right:{National Treatment on \\[1mm] \hspace{2mm} Internal Taxation}] at ([shift=({0:5 cm})]T1) (T2) {III:2};
        \node[state, label=right:{General Elimination of \\[1mm] \hspace{-2mm} Quantitative Restrictions}] at ([shift=({-30:5 cm})]T1) (T3) {XI:1};
        \node[state, label=right:{Fair Adminstration of \\[1mm] \hspace{-1mm} Laws and Regulations}] at ([shift=({30:5 cm})]T1) (T4) {X:3(a)};
        
        
        % \def\circledarrow#1#2#3{ % #1 Style, #2 Center, #3 Radius
        % \draw[#1,-, densely dotted] (#2) +(-50:#3) arc(-50:0:#3);
        % }
        % \circledarrow{ultra thick, gray!50}{T1}{2cm};
        % \circledarrow{ultra thick, gray!50}{T1}{2.1cm};

        \begin{scope}[every edge/.append style={->, double=black, draw=white}] % for directed edge, change "style={->, double=black, draw=white}]"
            \path
            (T1) edge node[above] {$0.092$} (T2)
            (T1) edge node[above, xshift=5pt] {$0.07$} (T3)
            (T1) edge node[above, yshift=5pt] {$0.108$} (T4);
            
            \path[->] (T1) edge[thin,->,densely dotted] node[above, xshift=5pt] {$w_{ij_{1}}$} +(1.95,-3.3);
            \path[->] (T1) edge[thin,->,densely dotted] node[above, xshift=10pt] {$w_{ij_{2}}$} +(0,-3.75);
            \path[->] (T1) edge[thin,->,densely dotted] node[above, xshift=10pt] {$w_{ij_{3}}$} +(-1.95,-3.3);


        \end{scope}


    \end{tikzpicture}


    \caption{\textbf{Example of Network of Legal Articles of WTO agreements: }}
    \label{fig:def-example}
\end{figure}


\begin{figure}
    \begin{tightcenter}
        Article I
    \end{tightcenter}
    \begin{tightcenter}
        \textit{General Most-Favoured-Nation Treatment}
    \end{tightcenter}
    1. With respect to customs duties and charges of any kind imposed on or in connection
    with importation or exportation or imposed on the international transfer of payments for
    imports or exports, and with respect to the method of levying such duties and charges, and
    with respect to all rules and formalities in connection with importation and exportation, and
    with respect to all matters referred to \textbf{in paragraphs 2 and 4 of Article III}, any advantage,
    favour, privilege or immunity granted by any contracting party to any product originating in
    or destined for any other country shall be accorded immediately and unconditionally to the
    like product originating in or destined for the territories of all other contracting parties...
    \caption{\textbf{Example of content of a legal article of the WTO agreement:} Article I:1 of GATT 1994}
    \label{fig:gatt_art1}
\end{figure}

% == THREE SUBSYSTEM == 
\begin{figure}
    \centering{
        \input{market_access_directed.tex}
    }
    \caption{{\bf Network of the Articles that Achieves \textit{Market Access}:}
        This figure demonstrates a network of articles of WTO agreements
        that cooperatively achieves \textit{Market access} principle of WTO.
        Tariff and Non-tariff barriers such as quantitative restriction, internal taxations
        and extra fees for crossing border can inhibit the chance of foreign goods to cross the borders.
        Therefore, these articles tend to work together to ensure the \textit{Market access} principle working properly.
        (Directions and weights of network edges are omitted for brevity.)
    }
    \label{fig:market-aceess_directed}
\end{figure}




\begin{figure}
  \begin{displayquote}[][]
    \begin{center}
      Article 7
    \end{center}
    \begin{center}
      Terms of Reference of Panels
    \end{center}
  
    1. Panels shall have the following terms of reference unless the parties to the dispute
    agree otherwise within 20 days from the establishment of the panel:
  
    \begin{displayquote}[][]
  
      ``To examine, in the light of {\bf the relevant provisions} in (name of the covered
      agreement(s) cited by the parties to the dispute), the matter referred to the DSB by
      (name of party) in document … and to make such findings as will assist the DSB in
      making the recommendations or in giving the rulings provided for in that/those
      agreement(s).''
        
    \end{displayquote}
  
    2. Panels shall address {\bf the relevant provisions} in any covered agreement or agreements
    cited by the parties to the dispute. \ldots
  \end{displayquote}
  \caption{\textbf{Article 7 of the Dispute Settlement Understanding (DSU): } 
  DSU provides a legal guidelines on how judicial boides of WTO shall adjudicate the requested disputes.
  It explictly requires judicial bodies to interweave relevant articles of the WTO agreements to clarify
  it's meaning, scope and interpretation.
  } 
  \label{fig:art7dsu}
\end{figure}




% \begin{displayquote}
%     \center{Article 7}

%     \center{Terms of Reference of Panels}

%     1. Panels shall have the following terms of reference unless the parties to the dispute
%     agree otherwise within 20 days from the establishment of the panel:
%     "To examine, in the light of the relevant provisions in (name of the covered
%     agreement(s) cited by the parties to the dispute), the matter referred to the DSB by
%     (name of party) in document … and to make such findings as will assist the DSB in
%     making the recommendations or in giving the rulings provided for in that/those
%     agreement(s)."
%     2. Panels shall address the relevant provisions in any covered agreement or agreements
%     cited by the parties to the dispute.
% \end{displayquote}
