To address this issue,
this paper maps
the regulatory system of WTO DSB
as a network of legal articles
of the WTO agreements as formally defined in Figure \ref{fig:def} and illustrated in Figure \ref{fig:def-example}. This is because the rules of the WTO agreements
% \footnote{\textit{See} Article 7 in Dispute Settlement Rules: Understanding on Rules and Procedures Governing the Settlement of Disputes, Marrakesh Agreement Establishing the World Trade Organization, Annex 2, 1869 U.N.T.S. 401, 33 I.L.M. 1226 (1994)} 
explicitly requires judicial bodies to address
relevant articles together to construct its jurisprudence (\textit{See} Figure \ref{fig:art7dsu}).
Upon this requirement, judicial bodies refer to
multiple articles of the WTO agreements together
to identify the complex legal identity of the trade policy that led to the dispute.
In addition to it, judicial bodies cite multiple articles together
to provide an authoritative interpretation of the rules of the WTO agreements (\textit{See} Figure \ref{subfig:a:condprob})
% \citep{oesch2003standards}.

% == THREE SUBSYSTEM == 
\begin{figure}
    \centering{
        \begin{tikzpicture}[>={Stealth[color=black]},shorten >=1pt,node distance=2cm,on grid,initial/.style={}]
  \node[state, label=right:Interpretation of Tariff Concession] (T1) {II:1};
  \node[state, label=left:General Elimination of Quantitative Restrictions] at ([shift=({240:3 cm})]T1) (T4) {XI:1};
  \node[state, label=right:Fair Adminstration of Laws and Regulations] at ([shift=({240:3 cm})]T4) (T5) {X:3(a)};
  \node[state, label=above:Non-discriminatory Administration of Quantitative Restrictions] at ([shift=({100:6 cm})]T4) (T7) {XIII:1};
  \node[state, label=right:National Treatment on Internal Taxation] at ([shift=({10:2.5 cm})]T4) (T6) {III:2};
  \node[state, label=above:Fees Connected with Importation and Exportation] at ([shift=({150:5.5 cm})]T4) (T8) {VIII};

  \begin{scope}[every edge/.append style={->, double=black, draw=white}] % for directed edge, change "style={->, double=black, draw=white}]"
    \path (T1)
    edge   (T4)
    edge   (T6);
    \path (T4)
    edge   (T5)
    edge   (T6)
    edge   (T7);
    \path (T5)
    edge   (T6);
    \path (T8)
    edge   (T7)
    edge   (T1)
    edge   (T6);

  \end{scope}
\end{tikzpicture}

% to draw the node's border w/ color, refer to https://tex.stackexchange.com/questions/438412/how-to-add-border-to-a-node
    }
    \caption{{\bf Network of the Articles that Achieves \textit{Market Access}:}
        This figure demonstrates a network of articles of WTO agreements
        that cooperatively achieves \textit{Market access} principle of WTO.
        Tariff and Non-tariff barriers such as quantitative restriction, internal taxations
        and extra fees for crossing border can inhibit the chance of foreign goods to access the foreign market.
        Therefore, these articles tend to work together to ensure the \textit{Market access} principle working properly.
        (Directions and weights of network edges are omitted for brevity.)
    }
    \label{fig:market-aceess_directed}
\end{figure}

\begin{figure}[ht]
    \[\text{Network of legal articles of WTO agreements is defined as}\] %\textit{directed weighted graph}}  G = (V, E, w) \]
    \[ \textit{directed weighted graph }G = (V, E, w) \]
    \[\text{ where } V = \{v \mid v\text{ is a legal article of WTO agreement}\}  \text{ , } \]
    \[\vec{E} = \{(v_i, v_j) \mid (v_i, v_j)\in V \times V)\} \text{ and } \]
    \[w : V \times V \to \Bbb R_{+} \text{ } s.t. \sum_{v_j\in V}{w(v_i, v_j)} = 1 \text{ } \forall v_i \in V \]%\text{ and }\]
    \[\text{ To simplify the notation, let } w_{ij} := w(v_i, v_j)\]
    % \[\text{(}G\text{ is simply called \textit{directed weighted graph)}}\]
    \caption{\textbf{Formal Definition of Network of Legal Articles of WTO agreements: }I define network of legal articles of WTO agreements
        as a directed weighted graph where the sum of all weights coming out of a node sum up to 1 as illustrated in Figure \ref{fig:def-example}}
    \label{fig:def}
\end{figure}

% % == DEF EXAMPLE == 
\begin{figure}[]
    \begin{subfigure}[b]{1\textwidth}
        \centering{
                \begin{tikzpicture}[
        >={Stealth[color=black]}
        ,shorten >=1pt,node distance=2cm
        ,on grid,initial/.style={}
        ,every label/.style={align=left}
        ]
        \linespread{2}
        \node[state, label=above:{National Treatment \\[1mm] \hspace{0mm} on Internal Taxation}] (T1) at (10, 0) {$v_j$ = III:2};

        \node[text width=5cm] at (8.5,0)
        {$\sum_{v_i\in V}{w_{ij}} = 1$};

        \draw[ultra thick, gray!50, dotted, ->] (13,1.8) arc (30:-170:3.5);
        \draw[ultra thick, gray!50, dotted, -] (13,1.8) arc (30:170:3.5);

        \node[state, label=right:{Interpretation of \\[1mm] \hspace{-1.5mm} Tariff Concession}] at ([shift=({0:5 cm})]T1) (T2) {$v_i$ = II:1(b)};
        \node[state, label=right:{General Elimination of \\[1mm] \hspace{-2mm} Quantitative Restrictions}] at ([shift=({-30:5 cm})]T1) (T3) {XI:1};
        \node[state, label=right:{Fair Adminstration of \\[1mm] \hspace{-1mm} Laws and Regulations}] at ([shift=({30:5 cm})]T1) (T4) {X:3(a)};
        
        \draw[red,arrows={[red]<-}, ultra thick] node[above, xshift=12.75cm] {$P(v_j|v_i) = 8\%$} (T1) -- (T2);

        \begin{scope}[every edge/.append style={<-}] % for directed edge, change "style={->, double=black, draw=white}]"
            \path
            % (T1) edge[red, ultra thick] node[above] {$0.092$} (T2)
            (T1) edge[double=black, draw=white] node[above, xshift=5pt] {$11\%$} (T3)
            (T1) edge[double=black, draw=white] node[above, yshift=5pt] {$9.7\%$} (T4);

            \path[->] (T1) edge[thin,<-,densely dotted] node[above, xshift=5pt] {$w_{i_{1}j}$} +(1.95,-3.3);
            \path[->] (T1) edge[thin,<-,densely dotted] node[above, xshift=10pt] {$w_{i_{2}j}$} +(0,-3.75);
            \path[->] (T1) edge[thin,<-,densely dotted] node[above, xshift=10pt] {$w_{i_{3}j}$} +(-1.95,-3.3);
        \end{scope}

    \end{tikzpicture}

% \begin{figure}[ht]
%     \centering
    
%     \caption{\textbf{Example of Network of Legal Articles of WTO agreements: }}
%     \label{fig:def-example}
% \end{figure}

        }
        \caption{\textbf{Illustrated edge weights of a source node Article II:1(b)}}
        \label{subfig:a:art2b}
    \end{subfigure}
    \vfill
    \begin{subfigure}[b]{1\textwidth}
        \centering{
            \begin{displayquote}[][]
    \begin{center}
    \end{center}
  
    \begin{displayquote}[][]
    ``The dictionary definition of the noun `excess' is `[t]he amount by which one number
        or quantity exceeds another'. More specifically, `in excess of' means `more than'. Thus,
        as a textual matter, a particular number or quantity is `in excess of' another number
        or quantity if it is greater, regardless of the extent to which it is greater. 
      \textbf{\textit{Looking at the context of Article II:1(b), first sentence, we note that Article III:2, first
      sentence, of the GATT 1994 is cast in very similar terms and in fact uses the phrase
      `in excess of'}}:\\
        \begin{displayquote}
        The products of the territory of any contracting party imported into the
      territory of any other contracting party shall not be subject … to internal
      taxes or other internal charges of any kind in excess of those applied … to
      like domestic products \ldots
        \end{displayquote}   
    \end{displayquote}  
  \end{displayquote}

            \caption{\textbf{Exerpt from the panel report for the \textit{Russia – Tariff Treatment} case:} Panel clarifies the meaning of the \textit{`in excess of'} in Article III:2 with an anology to the Article II:1(b).}
            \label{subfig:a:condprob}
        }
    \end{subfigure}
    \caption{\textbf{Illustration of Network of Legal Articles of WTO agreements: }Every directed edge weight is interpreted as the conditional probability $P(v_j|v_i)$ of how probably a source node $v_i$ constitutes a legal context to clarify the meaning of the target node $v_j$ among all target nodes $v_{j \neq i} \in V$. Above subfigure (a) represents how jurisprudence of Panel stated in (b) is represented as an edge weight where source node Article II:1(b) constitutes the legal context of the target node Article III:2 with the probability of $9.2\%$ among all possible target articles.}
    \label{fig:def-example}
\end{figure}

\begin{displayquote}[][]
  \begin{center}
    Article 7\\
    Terms of Reference of Panels
  \end{center}

  1. Panels shall have the following terms of reference unless the parties to the dispute
  agree otherwise within 20 days from the establishment of the panel:

  \begin{displayquote}[][]

    ``To examine, in the light of {\bf the relevant provisions} in (name of the covered
    agreement(s) cited by the parties to the dispute), the matter referred to the DSB by
    (name of party) in document … and to make such findings as will assist the DSB in
    making the recommendations or in giving the rulings provided for in that/those
    agreement(s).''
      
  \end{displayquote}

  2. Panels shall address {\bf the relevant provisions} in any covered agreement or agreements
  cited by the parties to the dispute.

  \ldots
\end{displayquote}


\begin{figure}[h]
    \begin{center}
        Article I
    \end{center}
    \begin{center}
        \textit{General Most-Favoured-Nation Treatment}
    \end{center}
    1. With respect to customs duties and charges of any kind imposed on or in connection
    with importation or exportation or imposed on the international transfer of payments for
    imports or exports, and with respect to the method of levying such duties and charges, and
    with respect to all rules and formalities in connection with importation and exportation, and
    with respect to all matters referred to in paragraphs 2 and 4 of Article III, any advantage,
    favour, privilege or immunity granted by any contracting party to any product originating in
    or destined for any other country shall be accorded immediately and unconditionally to the
    like product originating in or destined for the territories of all other contracting parties...
    \caption{\textbf{Example of content of a legal article of the WTO agreement:} Article I:1 of GATT 1994 that prohibits the discrimination among similar products (WTO DSB prefer to call \textit{like products}).}
    \label{fig:gatt_art1}
\end{figure}



% \begin{displayquote}
%     \center{Article 7}

%     \center{Terms of Reference of Panels}

%     1. Panels shall have the following terms of reference unless the parties to the dispute
%     agree otherwise within 20 days from the establishment of the panel:
%     "To examine, in the light of the relevant provisions in (name of the covered
%     agreement(s) cited by the parties to the dispute), the matter referred to the DSB by
%     (name of party) in document … and to make such findings as will assist the DSB in
%     making the recommendations or in giving the rulings provided for in that/those
%     agreement(s)."
%     2. Panels shall address the relevant provisions in any covered agreement or agreements
%     cited by the parties to the dispute.
% \end{displayquote}
