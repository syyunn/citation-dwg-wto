\begin{figure}[h]
    \centering
    \begin{tikzpicture}
    % Conv Runs Output-0
    \foreach \i in {0,...,1} {
        \draw [black] (\i,\yMinOut) -- (\i,\yMaxOut) node [below] at (\i,\yMinOut) {};
    }
    \foreach \i in {\yMinOut,...,\yMaxOut} {
        \draw [black] (0,\i) -- (1,\i) node [left] at (0,\i) {};
    }
    % Conv Runs Output-1
    \draw [black] (1.5,\yMinOut+0.5) -- (1.5,\yMaxOut+0.5) node [below] at (1.5,\yMinOut+0.5) {};
    \foreach \i in {\yMinOut,...,\yMaxOut} {
        \draw [black] (1,\i+0.5) -- (1.5,\i+0.5) node [left] at (0,\i+0.5) {};
    }
    \draw [black] (0.5,\yMaxOut+0.5) -- (1,\yMaxOut+0.5) node [left] at (0,\yMaxOut+0.5) {};
    \draw [black] (0.5,\yMaxOut) -- (0.5,\yMaxOut+0.5) node [below] at (0.5,\yMinOut+0.5) {};
    
    % Conv Runs Output-2
    \draw [black] (2,\yMinOut+1) -- (2,\yMaxOut+1) node [below] at (2,\yMinOut+1) {};
    \foreach \i in {\yMinOut,...,\yMaxOut} {
        \draw [black] (1.5,\i+1) -- (2,\i+1) node [left] {};
    }
    \draw [black] (1,\yMaxOut+1) -- (1.5,\yMaxOut+1) node [left] {};
    \draw [black] (1,\yMaxOut+0.5) -- (1,\yMaxOut+1) node [below] {};

    % Conv Runs Output-2
    \draw [black] (2,\yMinOut+1) -- (2,\yMaxOut+1) node [below] at (2,\yMinOut+1) {};
    \foreach \i in {\yMinOut,...,\yMaxOut} {
        \draw [black] (1.5,\i+1) -- (2,\i+1) node [left] {};
    }
    \draw [black] (1,\yMaxOut+1) -- (1.5,\yMaxOut+1) node [left] {};
    \draw [black] (1,\yMaxOut+0.5) -- (1,\yMaxOut+1) node [below] {};
    
    % Conv Runs Output-3
    \draw [red, very thick] (2.5,\yMinOut+1.5) -- (2.5,\yMaxOut+1.5) node [below] {};
    \foreach \i in {\yMinOut,...,\yMaxOut} {
        \draw [red, very thick] (2,\i+1.5) -- (2.5,\i+1.5) node [left] {};
    }
    \draw [red, very thick] (1.5,\yMaxOut+1.5) -- (2,\yMaxOut+1.5) node [left] {};
    \draw [red, very thick] (1.5,\yMaxOut+1) -- (1.5,\yMaxOut+1.5) node [below] {};
    
    %Axis ticks
    \node [right, rotate=90, font=\small] at (-0.5,1) {\textbf{Feature Map Length = $n-h+1$}};
    \node [right, rotate=40, font=\small] at (-1,8) {\textbf{$m$ number of filters}};
    \node [right, rotate=47.5, font=\small] at (9,3) {\textbf{$m$-dimensional vector }};

    %Max Pool Output
    \draw [black] (7,3) -- (11,7) node [below] {};
    \draw [black] (8,3) -- (12,7) node [below] {};

    \foreach \i in {3,...,7} {
        \draw [black] (\i+4,\i) -- (\i+5,\i) node [left] {};
    }
    
    %Blue on Output
    \draw [blue, very thick] (7,3) -- (8,3) node [left] {};
    \draw [blue, very thick] (8,4) -- (9,4) node [left] {};
    \draw [blue, very thick] (7,3) -- (8,4) node [below] {};
    \draw [blue, very thick] (8,3) -- (9,4) node [below] {};

    %Red on Output
    \draw [red, very thick] (10,6) -- (11,6) node [left] {};
    \draw [red, very thick] (11,7) -- (12,7) node [left] {};
    \draw [red, very thick] (10,6) -- (11,7) node [below] {};
    \draw [red, very thick] (11,6) -- (12,7) node [below] {};

    % Connection between Two Matrix
    \draw[dotted,blue, very thick] (1,8) -- (8,4);
    \draw[dotted,blue, very thick] (1,1) -- (7,3);

    \draw[dotted,red, very thick] (2.5,9.5) -- (11,7);
    \draw[dotted,red, very thick] (2.5,2.5) -- (10,6);
    
    % Grid 
    \draw [step=1.0,blue, very thick] (0,8) grid (1,1); % 3-filter

    \end{tikzpicture} 

    
    \label{fig:maxpool1d}      
    % \caption{\textbf{Conv1D:} $m$ number of $h$-sized filter runs over the $n \times k $ embedding matrix and produces $m \times (n-h+1)$ matrix}
    \caption{\textbf{MaxPool1D:} Filter out max value for all $m$ number of feature map outputs from $m$ different convolution filters. MaxPool1D produces $m$ dimensional vector as an output for collection of those filtered max values.} % actually feature map means "after" non-linear activation.
\end{figure}

% This  dispute  concerns  the  Continued  Dumping 