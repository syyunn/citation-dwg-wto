% % == DEF EXAMPLE == 
\begin{figure}[]
    \begin{subfigure}[b]{1\textwidth}
        \[\text{Let } \delta^d_{ij} \text{ is defined to be } 1 \text{ if } \{(v_i, v_j) \mid v_i, v_j \in V \text{ and } i \neq j \} \subset c_{d \in D} \ \text{ else } 0\]
        \[\text{ where } V, D \text{ and } c_d \text{ is defined as in Figure \ref{fig:def:set-of-cited-articles}}. \]
        \[\text{Then let \textit{co-citation matrix} } M = (m_{ij}) \in \Bbb{N}^{|V| \times |V|} \text{ s.t. } m_{ij} = \sum_{d \in D}\sum_{i,j \in V}\delta^d_{ij}\]
        \caption{\textbf{Formal Definition of Co-citation Matrix}}
        \label{subfig:co-cites:def}
    \end{subfigure}
    \vfill
    \begin{subfigure}[b]{1\textwidth}
        \centering{
            % \begin{figure}[h]
%     \centering
    \begin{tikzpicture}
        \foreach \i in {\xMin,...,5} {
            \draw [black] (\i,4) -- (\i,9) node [below] at (\i,4) {};
        }
        \foreach \i in {4,...,9} {
            \draw [black] (\xMin,\i) -- (5,\i) node [left] at (\xMin,\i) {};
        }

        % \node [left] at (0,0.5) {\textbf{XXVIII}};
        % \node [left] at (0,1.5) {\textbf{XXVI:6}};
        % \node [left] at (0,2.5) {\textbf{XXIV:5(b)}};
        % \node [left] at (0,3.5) {\textbf{XXIV:12}};
        \node [left] at (0,4.5) {\textbf{\vdots}};
        \node [left] at (0,5.5) {\textbf{II:1}};
        \node [left] at (0,6.5) {\textbf{II}};
        \node [left] at (0,7.5) {\textbf{I:1}};
        \node [left] at (0,8.5) {\textbf{I}};

        % \node [left] at (8.8, 9.25) {\textbf{XXVIII}};
        % \node [left] at (7.8, 9.25) {\textbf{XXVI:6}};
        % \node [left] at (6.8, 9.25) {\textbf{XXIV:5(b)}};
        % \node [left] at (5.8, 9.25) {\textbf{XXIV:12}};
        \node [left] at (4.8, 9.25) {\textbf{\ldots}};
        \node [left] at (3.8, 9.25) {\textbf{II:1}};
        \node [left] at (2.8, 9.25) {\textbf{II}};
        \node [left] at (1.85, 9.25) {\textbf{I:1}};
        \node [left] at (0.7,9.25) {\textbf{I}};

        \node [left] at (0.7,8.5) {\textbf{0}};
        \node [left] at (1.7,8.5) {\textbf{3}};
        \node [left] at (2.8,8.5) {\textbf{7}};
        \node [left] at (3.8,8.5) {\textbf{2}};
        
        \node [left] at (0.7,7.5) {\textbf{3}};
        \node [left] at (1.7,7.5) {\textbf{0}};
        \node [left] at (2.8,7.5) {\textbf{3}};
        \node [left] at (3.8,7.5) {\textbf{4}};

        \node [left] at (0.7,6.5) {\textbf{7}};
        \node [left] at (1.7,6.5) {\textbf{3}};
        \node [left] at (2.8,6.5) {\textbf{0}};
        \node [left] at (3.8,6.5) {\textbf{4}};

        \node [left] at (0.7,5.5) {\textbf{2}};
        \node [left] at (1.7,5.5) {\textbf{4}};
        \node [left] at (2.8,5.5) {\textbf{4}};
        \node [left] at (3.8,5.5) {\textbf{0}};



        % \node [left] at (5.5,9.5) {\textbf{Embedding Dimension = $k$}};
        % \node [right, rotate=-90, font=\small] at (6.5,7) {\textbf{Max Sequence Length = $n$}};


    % \draw [step=1.0,blue, very thick] (0.5,0.5) grid (5.5,4.5);
    % \draw [very thick, brown, step=1.0cm,xshift=-0.5cm, yshift=-0.5cm] (0.5,0.5) grid +(5.5,4.5);
    \end{tikzpicture} 
%     \label{fig:visualize-co-cites}      
%     \caption{\textbf{Definition of Co-citation Matrix:} This paper defines co-citation matrix $C$ as a $\mid V \mid \times \mid V \mid$ matrix where each element $c_{ij}:= \text{Counts of } $ is count of all co-occurrence $  \text{ } s.t. \text{ } \forall i,j \in \mid V \mid$ } 
% \end{figure}

% This  dispute  concerns  the  Continued  Dumping 
            \caption{\textbf{Illustration of Co-citation Matrix}}
            \label{subfig:co-cites:illus}
        }
    \end{subfigure}
    \label{fig:def-illus-co-cites}      
    \caption{\textbf{Formal Definition and Illustration of Co-citation Matrix:} This paper defines co-citation matrix $M$ as subfigure $(a)$ and it's illustrated as subfigure $(b)$ using the paper's dataset. Note that co-citation matrix is \textit{symmetric}, $m_{ij} = m_{ji} \text{ } \forall i,j \in V$.} 
\end{figure}