This paper emphasizes the necessity of using textual information 
to qualitatively fit the edge weights $w^*$ for the \textit{directed weighted graph} $G^*$ as defined in Figure \ref{fig:def}. 
One can simply consider a co-citation pattern between the articles of WTO agreements as a a regulatory system of WTO DSB, however,
it simply allocates a huge edge weight for frequently cited articles and fails to explain how articles interact to achieve main principles of WTO.

\begin{itemize}
    \item Visualize the co-citation matrix that explains the structure \& construction of the matrix
\end{itemize}

This failure is mainly due to the insufficient information in co-citation matrix. Members tend to cite the articles of 
the WTO agreements based on the complex characteristics of 
the trade policy that led to the dispute, however, co-citation pattern 


if a method 
do not consider a way to encompass the information
about the identiy of the trade policy at issue with 
each article's regulatory content written in its own article text,
it can only reflect how articles of WTO agreements are co-cited 
together rather than reflecting how articles of WTO agreements 
cooperates to regulate each charactersitic of complex trade policy that led to the dispute. 
