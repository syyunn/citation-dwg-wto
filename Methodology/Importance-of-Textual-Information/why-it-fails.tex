I emphasizes the necessity of using textual information
to qualitatively fit the edge weight matrix $W^*$ as defined in Section \ref{subsec:def}. 
Rather than using the textual data, one can simply model the co-citation matrix as $W^*$, which counts the co-occurrences of each article with other articles.
% pattern between the articles of WTO agreements that counts the co-occurrences of each article with other articles, however,
However, it simply allocates a large edge weight for frequently cited articles and fails to explain how articles systematically interact with other articles.
This failure is mainly due to the insufficient information in the co-citation matrix. Members cite the articles of
the WTO agreements based on the complex characteristics of
the trade policy that led to the dispute. 
However, the co-citation pattern omits this contextual information. To emphasize the necessity of using the textual information, I prepared two different matrices $W_{\text{co-cites}}$ and $W_{\text{text}}$ that are both following the definition of \textit{edge weight matrix} $W$ in Section \ref{subsec:def}.
$W_{\text{co-cites}}$ is calculated using the co-citation pattern between the articles of the WTO agreements as formally defined
%  \textit{Normalized Co-citation Matrix} 
in Figure %\ref{fig:def-illus-co-cites} and 
\ref{fig:def-illus-normal-co-cites}.
$W_{\text{text}}$ is the one fitted using the textual information and the way how it's fitted will be explained at the following bodies of this section, in particular in Section \ref{subsec:rf}.
Two heatmaps visualized in Figure \ref{fig:heatmap:edgeweights:compare} shows how sparse the $W_{\text{co-cites}}$ is compared to the $W_{\text{text}}$. This sparsity indirectly refers to the insufficient information
to qualitatively map the jurisprudence of WTO DSB.
% and I will introduce the failure of the weight fitting algorithm using \textit{Random Forest} in case of using this sparse matrix $W_{\text{co_cites}}$.
In contrast with it, if we fit the \textit{edge weight matrix} $W$ using the textual information, we get a more dense and informative matrix as visualized in Figure \ref{subfig:adj_dense}.
% and visualized them as a heatmap in Figure \ref{fig:heatmap:edgeweights:compare}.
% $W_{\text{co-cites}}$ is an edge weight matrix that is calculated only using the co-citation pattern between articles without textual information.
% We can compare two weight matrices of each network that maps the regulatory system of WTO DSB as a network of articles of WTO agreements in Figure \ref{fig:sparse_dense}.
\begin{figure}[ht]
    \begin{subfigure}[b]{1\textwidth}
        \[\text{Let } \delta^d_{ij} \text{ is defined to be } 1 \text{ if } \{(v_i, v_j) \mid v_i, v_j \in V \text{ and } i \neq j \} \subset c_{d \in D} \ \text{ else } 0\]
        \[\text{ where } V, D \text{ and } c_d \text{ is defined as in Figure \ref{fig:def:set-of-cited-articles}}. \]
        \[\text{Then let \textit{co-citation matrix} } M = (m_{ij}) \in \Bbb{N}^{|V| \times |V|} \text{ s.t. } m_{ij} = \sum_{d \in D}\sum_{i,j \in V}\delta^d_{ij}\]
        \caption{\textbf{Formal Definition of Co-citation Matrix}}
        \label{subfig:co-cites:def}
    \end{subfigure}
    \vfill
    \begin{subfigure}[b]{1\textwidth}
        \centering{
            % \begin{figure}[h]
%     \centering
    \begin{tikzpicture}
        \foreach \i in {\xMin,...,5} {
            \draw [black] (\i,4) -- (\i,9) node [below] at (\i,4) {};
        }
        \foreach \i in {4,...,9} {
            \draw [black] (\xMin,\i) -- (5,\i) node [left] at (\xMin,\i) {};
        }

        % \node [left] at (0,0.5) {\textbf{XXVIII}};
        % \node [left] at (0,1.5) {\textbf{XXVI:6}};
        % \node [left] at (0,2.5) {\textbf{XXIV:5(b)}};
        % \node [left] at (0,3.5) {\textbf{XXIV:12}};
        \node [left] at (0,4.5) {\textbf{\vdots}};
        \node [left] at (0,5.5) {\textbf{II:1}};
        \node [left] at (0,6.5) {\textbf{II}};
        \node [left] at (0,7.5) {\textbf{I:1}};
        \node [left] at (0,8.5) {\textbf{I}};

        % \node [left] at (8.8, 9.25) {\textbf{XXVIII}};
        % \node [left] at (7.8, 9.25) {\textbf{XXVI:6}};
        % \node [left] at (6.8, 9.25) {\textbf{XXIV:5(b)}};
        % \node [left] at (5.8, 9.25) {\textbf{XXIV:12}};
        \node [left] at (4.8, 9.25) {\textbf{\ldots}};
        \node [left] at (3.8, 9.25) {\textbf{II:1}};
        \node [left] at (2.8, 9.25) {\textbf{II}};
        \node [left] at (1.85, 9.25) {\textbf{I:1}};
        \node [left] at (0.7,9.25) {\textbf{I}};

        \node [left] at (0.7,8.5) {\textbf{0}};
        \node [left] at (1.7,8.5) {\textbf{3}};
        \node [left] at (2.8,8.5) {\textbf{7}};
        \node [left] at (3.8,8.5) {\textbf{2}};
        
        \node [left] at (0.7,7.5) {\textbf{3}};
        \node [left] at (1.7,7.5) {\textbf{0}};
        \node [left] at (2.8,7.5) {\textbf{3}};
        \node [left] at (3.8,7.5) {\textbf{4}};

        \node [left] at (0.7,6.5) {\textbf{7}};
        \node [left] at (1.7,6.5) {\textbf{3}};
        \node [left] at (2.8,6.5) {\textbf{0}};
        \node [left] at (3.8,6.5) {\textbf{4}};

        \node [left] at (0.7,5.5) {\textbf{2}};
        \node [left] at (1.7,5.5) {\textbf{4}};
        \node [left] at (2.8,5.5) {\textbf{4}};
        \node [left] at (3.8,5.5) {\textbf{0}};



        % \node [left] at (5.5,9.5) {\textbf{Embedding Dimension = $k$}};
        % \node [right, rotate=-90, font=\small] at (6.5,7) {\textbf{Max Sequence Length = $n$}};


    % \draw [step=1.0,blue, very thick] (0.5,0.5) grid (5.5,4.5);
    % \draw [very thick, brown, step=1.0cm,xshift=-0.5cm, yshift=-0.5cm] (0.5,0.5) grid +(5.5,4.5);
    \end{tikzpicture} 
%     \label{fig:visualize-co-cites}      
%     \caption{\textbf{Definition of Co-citation Matrix:} This paper defines co-citation matrix $C$ as a $\mid V \mid \times \mid V \mid$ matrix where each element $c_{ij}:= \text{Counts of } $ is count of all co-occurrence $  \text{ } s.t. \text{ } \forall i,j \in \mid V \mid$ } 
% \end{figure}

% This  dispute  concerns  the  Continued  Dumping 
            \caption{\textbf{Illustration of Co-citation Matrix}}
            \label{subfig:co-cites:illus}
        }
    \end{subfigure}
    \label{fig:def-illus-co-cites}      
    \caption{\textbf{Formal Definition and Illustration of Co-citation Matrix:} This paper defines co-citation matrix $M$ as subfigure $(a)$ and it's illustrated as subfigure $(b)$ using the paper's dataset. Note that co-citation matrix is \textit{symmetric}, $m_{ij} = m_{ji} \text{ } \forall i,j \in V$.} 
\end{figure}
\begin{figure}[]
    \begin{subfigure}[b]{1\textwidth}
        \[\text{For given $M$ defined in Figure \ref{subfig:co-cites:def},} \]
        \[\text{let \textit{normalized co-citation matrix} } N = (n_{ij}) \in \Bbb{R}^{|V| \times |V|} \text{ s.t. } n_{ij} = \frac{m_{ij}}{\sum_{j \in V}m_{ij}} \]
        \caption{\textbf{Formal Definition of Normalized Co-citation Matrix}}
        \label{subfig:co-cites:def:normal}
    \end{subfigure}
    \vfill
    \begin{subfigure}[b]{1\textwidth}
        \centering{
            % \begin{figure}[h]
%     \centering
    \begin{tikzpicture}
        \foreach \i in {\xMin,...,5} {
            \draw [black] (\i,4) -- (\i,9) node [below] at (\i,4) {};
        }
        \foreach \i in {4,...,9} {
            \draw [black] (\xMin,\i) -- (5,\i) node [left] at (\xMin,\i) {};
        }

        % \node [left] at (0,0.5) {\textbf{XXVIII}};
        % \node [left] at (0,1.5) {\textbf{XXVI:6}};
        % \node [left] at (0,2.5) {\textbf{XXIV:5(b)}};
        % \node [left] at (0,3.5) {\textbf{XXIV:12}};
        \node [left] at (0,4.5) {\textbf{\vdots}};
        \node [left] at (0,5.5) {\textbf{II:1}};
        \node [left] at (0,6.5) {\textbf{II}};
        \node [left] at (0,7.5) {\textbf{I:1}};
        \node [left] at (0,8.5) {\textbf{I}};

        % \node [left] at (8.8, 9.25) {\textbf{XXVIII}};
        % \node [left] at (7.8, 9.25) {\textbf{XXVI:6}};
        % \node [left] at (6.8, 9.25) {\textbf{XXIV:5(b)}};
        % \node [left] at (5.8, 9.25) {\textbf{XXIV:12}};
        \node [left] at (4.8, 9.25) {\textbf{\ldots}};
        \node [left] at (3.8, 9.25) {\textbf{II:1}};
        \node [left] at (2.8, 9.25) {\textbf{II}};
        \node [left] at (1.85, 9.25) {\textbf{I:1}};
        \node [left] at (0.7,9.25) {\textbf{I}};

        \node [left] at (0.7,8.5) {\textbf{0}};
        \node [left] at (1.7,8.5) {\textbf{3}};
        \node [left] at (2.8,8.5) {\textbf{7}};
        \node [left] at (3.8,8.5) {\textbf{2}};
        
        \node [left] at (0.7,7.5) {\textbf{3}};
        \node [left] at (1.7,7.5) {\textbf{0}};
        \node [left] at (2.8,7.5) {\textbf{3}};
        \node [left] at (3.8,7.5) {\textbf{4}};

        \node [left] at (0.7,6.5) {\textbf{7}};
        \node [left] at (1.7,6.5) {\textbf{3}};
        \node [left] at (2.8,6.5) {\textbf{0}};
        \node [left] at (3.8,6.5) {\textbf{4}};

        \node [left] at (0.7,5.5) {\textbf{2}};
        \node [left] at (1.7,5.5) {\textbf{4}};
        \node [left] at (2.8,5.5) {\textbf{4}};
        \node [left] at (3.8,5.5) {\textbf{0}};



        % \node [left] at (5.5,9.5) {\textbf{Embedding Dimension = $k$}};
        % \node [right, rotate=-90, font=\small] at (6.5,7) {\textbf{Max Sequence Length = $n$}};


    % \draw [step=1.0,blue, very thick] (0.5,0.5) grid (5.5,4.5);
    % \draw [very thick, brown, step=1.0cm,xshift=-0.5cm, yshift=-0.5cm] (0.5,0.5) grid +(5.5,4.5);
    \end{tikzpicture} 
%     \label{fig:visualize-co-cites}      
%     \caption{\textbf{Definition of Co-citation Matrix:} This paper defines co-citation matrix $C$ as a $\mid V \mid \times \mid V \mid$ matrix where each element $c_{ij}:= \text{Counts of } $ is count of all co-occurrence $  \text{ } s.t. \text{ } \forall i,j \in \mid V \mid$ } 
% \end{figure}

% This  dispute  concerns  the  Continued  Dumping 
            \caption{\textbf{Illustration of Noramlized Co-citation Matrix}}
            \label{subfig:co-cites:illus:normal}
        }
    \end{subfigure}
    \caption{\textbf{Formal Definition and Illustration of Normalized Co-citation Matrix:} This paper defines normalized co-citation matrix $N$ of $M$ as subfigure $(a)$ and it's illustrated as subfigure $(b)$ using the paper's dataset. Note that normalized co-citation matrix is no more \textit{symmetric}, $n_{ij} \neq n_{ji} \text{ } \forall i,j \in V$. This definition is prepared to fit the definition of co-citation matrix to that of $W$ in Figure \ref{fig:def}.} 
    \label{fig:def-illus-normal-co-cites}
\end{figure}
Upon this observation, this paper justifies the use of a deep neural network to process information embedded in the text description of the dispute and regulatory content of the articles. 
This is because the deep neural network is known to effectively
extract information from the textual data to perform various tasks such as text classification \citep{minaee2020deep}, text summarization \citep{textsum} and text generation \citep{guo2017long}.
% This paper will compare two different results that one uses textual information using deep learning and the other one
% uses the frequency of co-citation only after explaining how deep learning works to encompass the textual information.
% if a method
% do not consider a way to encompass the information
% about the identiy of the trade policy at issue with
% each article's regulatory content written in its own article text,
% it can only reflect how articles of WTO agreements are co-cited
% together rather than reflecting how articles of WTO agreements
% cooperates to regulate each charactersitic of complex trade policy that led to the dispute.
 
 
 

