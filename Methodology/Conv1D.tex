\begin{figure}[ht]
    \centering
    \begin{tikzpicture}
        % Conv Runs
        \foreach \i in {\xMin,...,\xMax} {
            \draw [black] (\i,\yMin) -- (\i,\yMax) node [below] at (\i,\yMin) {};
        }
        \foreach \i in {\yMin,...,\yMax} {
            \draw [black] (\xMin,\i) -- (\xMax,\i) node [left] at (\xMin,\i) {};
        }

        \node [left] at (0,0.5) {\textbf{[PAD]}};
        \node [left] at (0,1.5) {\textbf{thereafter}};
        \node [left] at (0,2.5) {\textbf{issued}};
        \node [left] at (0,3.5) {\textbf{or}};
        \node [left] at (0,4.5) {\textbf{\vdots}};
        \node [left] at (0,5.5) {\textbf{the}};
        \node [left] at (0,6.5) {\textbf{concerns}};
        \node [left] at (0,7.5) {\textbf{dispute}};
        \node [left] at (0,8.5) {\textbf{This}};

        \node [left] at (5.5,9.5) {\textbf{Embedding Dimension = $k$}};
        \node [right, rotate=-90, font=\small] at (6.5,7) {\textbf{Max Token Length = $n$}};


    \draw [step=1.0,blue, very thick] (0,9) grid (6,6); % 3-filter
    \node [right, blue, font=\huge, rotate=-90] at (2.8,5.3) {\textbf{$\dasharrow$}};
    \draw [step=1.0,blue, very thick] (0,0) grid (6,3); % 3-filter

    % Conv Runs Output
    \foreach \i in {\xMinOut,...,\xMaxOut} {
        \draw [black] (\i,\yMinOut) -- (\i,\yMaxOut) node [below] at (\i,\yMinOut) {};
    }
    \foreach \i in {\yMinOut,...,\yMaxOut} {
        \draw [black] (\xMinOut,\i) -- (\xMaxOut,\i) node [left] at (\xMinOut,\i) {};
    }
    \node [right, rotate=-90, font=\small] at (11.5,8) {\textbf{Feature Map Length = $n-h+1$}};

    \draw [step=1.0,blue, very thick] (11,8) grid (10,7); % 3-filter
    \draw [step=1.0,blue, very thick] (11,2) grid (10,1); % 3-filter

    
    % Connection between Two Matrix
    \draw[dotted,blue, very thick] (6,9) -- (10,8);
    \draw[dotted,blue, very thick] (6,6) -- (10,7);

    \draw[dotted,blue, very thick] (6,0) -- (10,1);
    \draw[dotted,blue, very thick] (6,3) -- (10,2);

    \end{tikzpicture} 

    
    % \caption{\textbf{Conv1D:} $m$ number of $h$-sized filter runs over the $n \times k $ embedding matrix and produces $m \times (n-h+1)$ matrix}
    \caption{\textbf{Conv1D:} $h$-sized filter runs over the $n \times k $ embedding matrix and produces $(n-h+1)$ size of feature map.} % actually feature map means "after" non-linear activation.
    \label{fig:conv1d}      

\end{figure}

% This  dispute  concerns  the  Continued  Dumping 