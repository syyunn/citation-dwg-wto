% WTO DSB is a case law system. 
% Therefore the system evolves as members use the system.
Upon the understanding that articles are cohesively working together to 
identify the legal characteristic of a trade policy or to achieve main principles of WTO, 
this paper proposed a new method that can materialize the relationship between articles 
in a form of weighted network using deep and machine learning. 

As shown in the section \ref{ef}, the result network captures the important interactions 
between articles of WTO agreements that are covered by the \textit{Panel} and \textit{Appellate Body} with relatively large edge weights.
% Since those interactions are also covered by the jurisprudences of the \textit{Panel} and \textit{Appellate Body},
% we can understand this networ captures the important interactions with 
As legal experts understand the important cohesion between articles that constitutes important jurisprudence,
this method can also provide a general guideline about where to focus and how to mingle the articles of WTO agreements to justly claim 
the illegality of given trade policy and one's impaired benfit. Moreover, though jurisprudence of WTO DSB constantly 
evolves with new cases, this method also can evolve by training new cases.

Therefore, this method could be considered as an technical solution to narrow the gap between developing and developed countries in terms of legal capacity in WTO DSB.
Without spending lots of time to study and elaborate the jurisprudence explained in the \textit{Panel} and \textit{Appellate Body} report, this method provide a legal guidlines where to focus and robustly adapts to the new cases.
Previous approach that provides legal advice to developing countries is ineffective because it cannot create a shared understanding over the system between developing and developed countries.
If we shift our focus to these technical solutions that can materialize the current shape of the system, WTO
will become more effective as members can discuss their trade issues upon the measurable ground of shared understanding about how WTO works.

% to narrow this gap by unilaterally 


% and this was the main cause of the widening gap of legal capacity between those two parties, however, t



% could be adopted as an technical solution to narrow the gap between 
% developing and developed countries in terms of legal capacity in WTO DSB. 



% Regulatory system of WTO DSB is notrorious for its complexity because it is desgined to evolve with the actual use cases.  