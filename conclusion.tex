Upon the understanding that articles are interactively working together to
constitute the jurisprudence of WTO DSB,
this paper proposed a new method that can materialize the relationship between articles
in a form of the weighted and directed network using deep and machine learning.
 
As shown in section \ref{ef}, the fitted network captures the important interactions
between articles of WTO agreements that are covered by the Panel and Appellate Body with relatively large edge weights.
As legal experts also understand those cohesions between articles as important jurisprudence of WTO DSB,
this method can also provide a general guideline about where to focus and how to mingle the articles of WTO agreements to justly claim
the illegality of a trade policy at issue. 
Moreover, though jurisprudence of WTO DSB constantly
evolves with new cases, this method can adapt to the newly created jurisprudences by training new cases.
 
Therefore, this method could be considered as a technical solution to narrow the gap between developing and developed countries in terms of legal capacity in WTO DSB.
Without spending lots of time to study and elaborate the jurisprudences as explained in the Panel and Appellate Body report, this method provides legal guidelines on where to focus and how to compose the legal claim with which articles.
Moreover, the previous approach to provide legal advice to developing countries from WTO has been ineffective because it cannot create a shared understanding of the system between developing and developed countries.
If we shift our focus to these technical solutions that can materialize the current shape of the system, WTO DSB
will become more effective as members can discuss their trade issues upon the measurable ground of the shared understanding about how WTO DSB works.

