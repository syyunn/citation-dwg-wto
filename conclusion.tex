% WTO DSB is a case law system. 
% Therefore the system evolves as members use the system.
Upon the understanding that articles are interactively working together to 
constitute the jurisprudences of WTO DSB, 
this paper proposed a new method that can materialize the relationship between articles 
in a form of weighted and directed network using deep and machine learning. 

As shown in the section \ref{ef}, the result network captures the important interactions 
between articles of WTO agreements that are covered by the \textit{Panel} and \textit{Appellate Body} with relatively large edge weights.
% Since those interactions are also covered by the jurisprudences of the \textit{Panel} and \textit{Appellate Body},
% we can understand this networ captures the important interactions with 
As legal experts understand those important cohesions between articles that constitute the important jurisprudences of WTO DSB,
this method can also provide a general guideline about where to focus and how to mingle the articles of WTO agreements to justly claim 
the illegality of a trade policy at issue and satisfactory adjustment for one's impaired benfit. Moreover, though jurisprudence of WTO DSB constantly 
evolves with new cases, this method can adapt to the newly created jurisprudences by training new cases.

Therefore, this method could be considered as an technical solution to narrow the gap between developing and developed countries in terms of legal capacity in WTO DSB.
Without spending lots of time to study and elaborate the jurisprudences explained in the \textit{Panel} and \textit{Appellate Body} report, this method provide a legal guidlines where to focus and how to compose the legal claim with which articles that fits to the case at issue.
Moreover, previous approach that provided legal advices to developing countries has been ineffective because it cannot create a shared understanding over the system between developing and developed countries.
If we shift our focus to these technical solutions that can materialize the current shape of the system, WTO DSB
will become more effective as members can discuss their trade issues upon the measurable ground of shared understanding about how WTO DSB works.

% to narrow this gap by unilaterally 


% and this was the main cause of the widening gap of legal capacity between those two parties, however, t



% could be adopted as an technical solution to narrow the gap between 
% developing and developed countries in terms of legal capacity in WTO DSB. 



% Regulatory system of WTO DSB is notrorious for its complexity because it is desgined to evolve with the actual use cases.  