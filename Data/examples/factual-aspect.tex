Excerpt below is from the Panel report for the 
\textit{US - Offset Act (Byrd Amendment)}
\footnote{Panel Report, United States — Continued Dumping and Subsidy Offset Act of 2000, WTO Doc. WT/DS217/R (adopted Jan. 27, 2003).} case.\\

\begin{tcolorbox}[breakable]
\noindent{\bf II. FACTUAL ASPECTS}\\\\
2.1 \quad This dispute concerns the Continued Dumping and Subsidy Offset Act of 2000 (the
“CDSOA” or the “Offset Act”), which was enacted on 28 October 2000 as part of the Agriculture,
Rural Development, Food and Drug Administration and Related Agencies Appropriations Act, 2001.1
The CDSOA amends Title VII of the Tariff Act of 1930 by adding a new section 754 entitled
Continued Dumping and Subsidy Offset. Regulations prescribing administrative procedures under the
Act were brought into effect on September 21, 2001.\\

\noindent2.2 \quad The CDSOA provides that :

\blockquote{
    Duties assessed pursuant to a countervailing duty order, an anti-dumping duty order,
    or a finding under the Antidumping Act of 1921 shall be distributed on an annual
    basis under this section to the affected domestic producers for qualifying
    expenditures. Such distribution shall be known as “the continued dumping and
    subsidy offset”.
    }
\\\\
\noindent2.3 \quad The term “affected domestic producers” means :

\blockquote{
    a manufacturer, producer, farmer, rancher, or worker representative (including
associations of such persons) that – \\\\
        (A) was a petitioner or interested party in support of the petition with respect
to which an anti-dumping duty order, a finding under the Antidumping Act of 1921,
or a countervailing duty order has been entered, and \\\\
\quad \quad (B) remains in operation. \\\\
Companies, business, or persons that have ceased the production of the product
covered by the order or finding or who have been acquired by a company or business
that is related to a company that opposed the investigation shall not be an affected
domestic producer.
}
\\\\
\noindent 2.4 \quad In turn, the term “qualifying expenditure” is defined by the CDSOA as “expenditure[s]
incurred after the issuance of the anti-dumping duty finding or order or countervailing duty order in
any of the following categories:
\blockquote{\\
(A) Manufacturing facilities.\\
(B) Equipment.\\
(C) Research and development.\\
(D) Personnel training.\\
(E) Acquisition of technology.\\
(F) Health care benefits to employees paid for by the employer.\\
(G) Pension benefits to employees paid for by the employer.\\
(H) Environmental equipment, training or technology.\\
(I) Acquisition of raw materials and other inputs.\\
(J) Working capital or other funds needed to maintain production.”
}
\\\\
\noindent 2.5 \quad The CDSOA provides that the Commissioner of Customs shall establish in the Treasury of
the United States a special account with respect to each order or finding8
 and deposit into such
account all the duties assessed under that Order.9
 The Commissioner of Customs shall distribute all
funds (including all interest earned on the funds) from the assessed duties received in the preceding
fiscal year to affected domestic producers based on a certification by the affected domestic producer
that he is eligible to receive the distribution and desires to receive a distribution for qualifying
expenditures incurred since the issuance of the order or finding.10 Funds deposited in each special
account during each fiscal year are to be distributed no later than 60 days after the beginning of the
following fiscal year.11 The CDSOA and regulations prescribe that (1) if the total amount of the
certified net claims filed by affected domestic producers does not exceed the amount of the offset
available, the certified net claim for each affected domestic producer will be paid in full, and (2) if the
certified net claims exceed the amount available, the offset will be made on a pro rata basis based on
each affected domestic producer’s total certified claim.\\

\noindent 2.6 \quad Special accounts are to be terminated after “(A) the order or finding with respect to which the
account was established has terminated; (B) all entries relating to the order or finding are liquidated
and duties assessed collected; (C) the Commissioner has provided notice and a final opportunity to
obtain distribution pursuant to subsection (c); and (D) 90 days has elapsed from the date of the notice
described in subparagraph (C).” All amounts that remain unclaimed in the Account are to be
permanently deposited into the general fund in the US Treasury.12\\

\noindent 2.7 \quad The CDSOA applies with respect to all anti-dumping and countervailing duty assessments
made on or after 1 October 200013 pursuant to an anti-dumping order or a countervailing order or a
finding under the Antidumping Act of 1921 in effect on 1 January 1999 or issued thereafter. [END]

\end{tcolorbox}