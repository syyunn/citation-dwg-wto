Every lawsuit in WTO DSB
cites multiple set of articles
as shown in Table
\ref{xltabular:cited-article-for-us-offset}.
To collect this set of articles claimed for the same dispute, I wrote a program that collects this set of articles cited for the same dispute from the WTO official webpage\footnote{\url{https://www.wto.org/english/tratop_e/dispu_e/dispu_status_e.htm}}.
The webpage chronologically lists up all dispute cases
requested to WTO DSB and the program visits each page of 143 cases
and collects the cited articles. Among all the agreements included in the WTO agreements\footnote{
   WTO agreements is comprised of multiple agreements, such as
   General Agreement on Tariffs and Trade 1994,
   Agreement on Agriculture,
   Agreement on the Application of Sanitary and Phytosanitary Measures,
   Agreement on Textiles and Clothing,
   Agreement on Technical Barriers to Trade,
   Agreement on Trade-Related Investment Measures,
   Agreement on Implementation of Article VI of the General Agreement on Tariffs and Trade 1994 (antidumping),
   Agreement on Subsidies and Countervailing Measures,
   Agreement on Rules of Origin,
   Agreement on Safeguards and so on.
   } as a component,
this paper collected articles from \textbf{General Agreement on Tariffs and Trade 1994 (GATT 1994)} only.
This is because articles in GATT 1994 constitutes basic set of trade rules of WTO and other agreements
elaborates the articles of GATT 1994 more in detail \citep{world1999wto}. For example, the official name of \textit{Agreement on Anti-dumping}
is \textit{Agreement on Implementation of \textbf{Article VI of the GATT 1994}}
where the name self-explains that it elaborates on the article VI of GATT 1994.
The collected result is listed in the \hyperref[sub:cited-articles-table]{Appendix A.2}. Figure \ref{fig:set-of-articles-used}
lists up 80 different articles of GATT 1994 cited in 143 cases without duplication.
 
 
\begin{figure}[h]
   \begin{quote}
   I,
   I:1,
   II,
   II:1,
   II:1(a),
   II:1(b),
   II:2,
   II:3,
   III,
   III:1,
   III:2,
   III:4,
   III:5,
   III:7,
   IV,
   IX,
   IX:2,
   V,
   V:1,
   V:2,
   V:3,
   V:3(a),
   V:4,
   V:5,
   V:6,
   V:7,
   VI,
   VI:1,
   VI:1(a),
   VI:1(b),
   VI:2,
   VI:3,
   VI:5(a),
   VI:6,
   VII,
   VII:1,
   VII:2,
   VII:5,
   VIII,
   VIII:1,
   VIII:3,
   VIII:4,
   X,
   X:1,
   X:2,
   X:3,
   X:3(a),
   XI,
   XI:1,
   XIII,
   XIII:1,
   XIII:2,
   XIII:3(b),
   XIX,
   XIX:1,
   XIX:2,
   XIX:3,
   XV,
   XVI,
   XVI:1,
   XVI:4,
   XVII,
   XVII:1,
   XVII:1(c),
   XVIII,
   XVIII:10,
   XVIII:11,
   XX,
   XXI,
   XXII,
   XXII:1,
   XXIII,
   XXIII:1,
   XXIII:1(a),
   XXIII:1(b),
   XXIV,
   XXIV:12,
   XXIV:5(b),
   XXIV:6,
   XXVIII
   \end{quote}
   \caption{
       \textbf{
           Set of articles of GATT 1994 collected and used in this paper:
           }These articles comprises the node set $V$ and their ordered pairs comprise the edge set $E$ in Figure \ref{fig:def}
       }
   \label{fig:set-of-articles-used}
\end{figure}
 
\begin{figure}[ht]
   \[\text{Let } D \text{ is a set of DS case numbers listed in Figure \ref{fig:ds-cases-used}.} \] %\textit{directed weighted graph}}  G = (V, E, w) \]
   \[\text{Then there exists } c_d = \{v_d \in V \mid\ v_d \text{ is an article cited in the case } d \in D\} \]
   \[\text{ where } V \text{ is set of articles listed in Figure \ref{fig:set-of-articles-used}}.\]
   \[\text{Then define set of cited articles } C = \{c_d \mid d \in D\}\]
   \caption{\textbf{Formal Definition of Set of Cited Articles: }I formally define a set of cited articles $C$ and the elements of $C$ are listed in \hyperref[sub:cited-articles-table]{Appendix A.2}.}
   \label{fig:def:set-of-cited-articles}
\end{figure}
 
 
\subsubsection{Various Levels of Scope in Cited Articles}
As shown in Figure \ref{fig:set-of-articles-used},
members sometimes
cite articles in different levels of scope. For example,
For the Article II, member sometimes cites
Article II as a whole but sometimes cites
Article II:1 or Article II:1(a).
This is because two main judicial bodies of WTO DSB, \textit{Panel and Appellate Body},
both constitute its legal precedents citing articles of the WTO agreements in
various levels of scope.
Those judicial bodies cite the articles with the level of various scopes, such as \textit{Title, Article, Paragraph, Sentence} or \textit{Term} as shown in Table {\ref{xltabular:level-of-scopes}}.
Following this jurisprudence, members also cite articles in different levels of scope to
make their legal claim fit and valid according to the current jurisprudences of WTO DSB.\\ %as much as possible.\\
 
% \clearpage
 
\begin{xltabular}{\linewidth}{lXX}
   \caption{\textbf{Various Levels of Scope Adopted to Cite Articles of WTO agreemnts}}\\
   \hline
   \textbf{\normalsize Scope}
   & \textbf{\normalsize Quote}
   & \textbf{\normalsize Source}
   \\
   \endfirsthead
   \hline \hline
   Title
   & ``As the \textbf{\textit{title}
   of Article 21 makes clear},
   the task of panels \ldots
   forms part of the process
   of the `Surveillance of
   Implementation of the
   Recommendations and Rulings' of the
   DSB. \ldots''
   & Appellate Body Report, \textit{US – Shrimp (Malaysia)}, paras. 86-87.
   \\
   \hline
   Article
   &  ``The sequence of steps indicated above in the analysis of a claim of justification under \textbf{Article XX} reflects, not inadvertence or random choice, but rather the fundamental structure and logic of Article XX. \ldots''
   & Appellate Body Report, \textit{US – Shrimp (Malaysia)}, paras. 119-120.
   \\
   \hline
   Paragraph
   &  ``The verb `may' in \textbf{Article VI:2} of the GATT 1994 is, in our opinion, properly
   understood as giving Members a choice between imposing an anti-dumping duty or
   not, as well as a choice between imposing an anti-dumping duty equal to the dumping
   margin or imposing a lower duty. \ldots''
   & Appellate Body Report, \textit{US – 1916 Act}, paras. 116.    
   \\
   \hline
   Sentence
   & ``The customary rules of interpretation of public international law as
   required by \textbf{the first sentence of Article 17.6(ii) of the Anti-Dumping Agreement}, do
   not admit of another interpretation as far as the issue of zeroing raised in this appeal
   is concerned.''
   & Appellate Body Report, \textit{US – Zeroing (EC)}, paras. 132-133.
   \\
   \hline
   Term
   & ``Article II:1(a) provides that a
   Member shall accord to the `commerce' of other Members treatment no less
   favourable than that provided for in its Schedule. \textbf{The term `commerce'} is defined as
   referring broadly to the exchange of goods such that, in this provision, the 'commerce'
   of a Member should be understood to refer to all such exchanges of that Member''
   & Appellate Body Report, \textit{Colombia – Textiles}, para. 5.34.
   \\
   \hline
   \label{xltabular:level-of-scopes}
\end{xltabular}
 
 
% \begin{figure}[h]
% \begin{quote}
%     \ldots As the \textit{title} of Article 21 makes clear, the task of panels \ldots
%      forms part of the process
%     of the 'Surveillance of Implementation of the Recommendations and Rulings' of the
%     DSB. - \textit{Appellate Body Report, US – Shrimp (Malaysia), paras. 86-87. }
% \end{quote}
% \caption{Various Interpretative Scopes of Panel and Appellate Body}
% \end{figure}
 
% paragraph is also important in terms of scope.
 
 
 

