As explained in the introduction,
a trade policy that led to a dispute (formally called as \textit{Government Measure}) is pretty much complicated.
For example, the Panel for the \textit{US - Offset (Byrd Amendment)} explains:

\noindent To address this complexity, 
members who raise the claim (formally called \textit{complainant}) usually cite multiple articles of the WTO agreements. For example, in the
\textit{US - Offset (Byrd Amendment)} case,
a group of complainants\footnote{Australia,
    Brazil,
    Chile,
    European Communities,
    India,
    Indonesia,
    Japan,
    Korea and Thailand}
cited articles as shown in Table \ref{xltabular:cited-article-for-us-offset} from the WTO agreements to claim possible inconsitency of CDSOA to these articles\footnote{It is worth noting that the WTO agreements comprises many different agreements covering each specific topic in trade such as \textit{Agreement on Anti-dumping, Agreement on Subsidies and Countervailing Measures} and so on.}:\\
\begin{xltabular}{\linewidth}{ l | X }
    % \caption{explain something if needed}
    % \\
    \hline

    \textbf{\normalsize Name of WTO Agreement} & \textbf{\normalsize Cited Articles}\\
    \endfirsthead
    \hline \hline

    Agreement on Anti-dumping& 1, 5.4, 8, 18.1, 18.4 \\ \hline
    General Agreement on Tariffs and Trade 1994& VI:3, X:3, XXIII:1, VI:2  \\ \hline
    Agreement on Subsidies and Countervailing Measures& 4.10, 7.9, 10, 11.4, 18, 32.1, 32.5 \\ \hline
    Agreement Establishing the World Trade Organization & XVI:4 \\ \hline
    \caption{Cited Aticles in \textit{US - Offset (Byrd Amendment)}}
    \label{xltabular:cited-article-for-us-offset}
\end{xltabular}


\noindent Upon this understanding, 
I collected two different types of data for 143 different dispute cases requested to WTO DSB.\footnote{List of collected cases is
    available at \hyperref[sub:cited-articles-table]{Appendix A.2}
}.
One is textual description of trade policy
that led to the dispute \hyperref[sub:factual-aspect-example]{(\textit{See} an example at Appendix A.2)} and the other one is
set of articles of the WTO agreements that are
cited for each dispute \hyperref[sub:cited-articles-table]{(Appendix A.3)}.
I will explain the format and the content of
data with an example at the following subsections.
% Technical details about the automated way 
% of collecting data will be
% explained in the \hyperref[appendix:panel-report-toc]{Appendix A.4}.

\begin{figure}[h]
    \centering
    \includegraphics[scale=0.3]{Data/pngs/panel_report_toc.png}
    \caption{{\bf Table of Contents of the Panel Report}}
    % : }It shows {\bf II. FACTUAL ASPECTS}
\end{figure}

% \begin{center}
%     \includegraphics[scale=0.3]{Data/pngs/panel_report_toc.png}
% \end{center}


% This section explains how I collected 
% which data to train the neural network. 
% Basically, 





% I collected two different 
% types of data, one is textual description of trade policy 
% that led to the dispute and the other one is 
% set of articles of the WTO agreement that are
% cited for the dispute.s



% \begin{displayquote}
%     ``Korea' s domestic support for beef in 1997 and 1998 exceeded the de minimis level contrary to Article 6 of the Agreement on Agriculture.''
%   \end{displayquote}

%   \begin{itemize}
%     \item exemplify as detail as possible to inform readers about how data looks like.
%     \item Privide a running example that shows how WTO works with data. 
%     \item (Borrow from the previous paper)

%   \end{itemize}
