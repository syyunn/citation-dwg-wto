\begin{table}[t!]
   \setlength\tabcolsep{15pt}
   \begin{tabular}{ c | c }
       \hline
       \textbf{\normalsize Name of WTO Agreement}          & \textbf{\normalsize Cited Articles} \\
       \hline \hline
       Agreement on Anti-dumping                           & 1, 5.4, 8, 18.1, 18.4               \\ \hline
       General Agreement on Tariffs and Trade 1994         & VI:3, X:3, XXIII:1, VI:2            \\ \hline
       Agreement on Subsidies and Countervailing Measures  & 4.10, 7.9, 10, 11.4, 18, 32.1, 32.5 \\ \hline
       Agreement Establishing the World Trade Organization & XVI:4                               \\ \hline
   \end{tabular}
   \caption{\textbf{Cited articles in \textit{US - Offset (Byrd Amendment)} by complainants}}
   \label{xltabular:cited-article-for-us-offset}
\end{table}

Members who raise the claim (preferably called \textit{complainant} in WTO DSB) usually cite multiple articles of the WTO agreements. This is to cover the complex characteristics of a trade policy that led to the dispute.
% As explained in the introduction,
% a trade policy that led to a dispute (preferably called as \textit{Government Measure} in WTO DSB) is pretty much complicated as explicitly expressed by the Panel in Figure \ref{fig:complex-measure}.
% To address this complexity,
% members who raise the claim (preferably called \textit{complainant} in WTO DSB) usually cite multiple articles of the WTO agreements at the same time. 
For example, in the
\textit{US - Offset} case,
a group of complainants\footnote{Australia,
   Brazil,
   Chile,
   European Communities,
   India,
   Indonesia,
   Japan,
   Korea and Thailand}
cited articles as shown in Table \ref{xltabular:cited-article-for-us-offset} from the WTO agreements to claim inconsistencies of \textit{Continued Dumping and Subsidy Act of 2000} (CDSOA) of the United States to those cited articles\footnote{It is worth noting that the WTO agreements comprise many different agreements covering each specific topic in trade such as \textit{Agreement on Anti-dumping, Agreement on Subsidies and Countervailing Measures, Agreement on Agriculture} and so on.}.

Upon this understanding of multiple citations,
I collected two different types of data. %for 143 different disputes requested to the WTO DSB\footnote{\textit{Listed} case numbers at \hyperref[sub:cited-articles-table]{Appendix A.2}.}. 
One is textual description of the dispute\footnote{\hyperref[sub:factual-aspect-example]{\textit{Check} the CDSOA example at Appendix A.1}} and the other one is
set of articles of the WTO agreements that are
cited for each dispute\footnote{\hyperref[sub:cited-articles-table]{\textit{See} Appendix A.3}}.
I will explain each type of data in the following subsections in detail.
% I will explain the data source, structure, and collection method for two different types of data at the following subsections.\\


 
\begin{figure}[h]
    \centering
    \includegraphics[scale=0.3]{Data/pngs/panel_report_toc.png}
    \caption{{\bf Table of Contents of the Panel Report}}
    % : }It shows {\bf II. FACTUAL ASPECTS}
\end{figure}

% \begin{center}
%     \includegraphics[scale=0.3]{Data/pngs/panel_report_toc.png}
% \end{center}

 
 

